
\makeglossaries

\newglossaryentry{differentia}
{
    name=differentia,
    description={randomly generated information that can be used to differentiate two strata at the same layer (with a probability of spurious collision)}
}

\newglossaryentry{stratum}
{
    name=stratum,
    description={differentia container associated with a particular historical stage}
}

\newglossaryentry{deposit}
{
    name=deposit,
    description={the act of extending the historical record with a new stratum}
}

\newglossaryentry{deposition}
{
    name=deposition,
    description={the stratum appended to the historical record}
}

\newglossaryentry{time point}
{
    name=time point,
    description={refers to the stage where a specific number of depositions have taken place}
}

\newglossaryentry{deposition time}
{
    name=deposition time,
    description={refers to the time point at which a stratum was deposited, zero indexed}
}

\newglossaryentry{genesis}
{
    name=genesis,
    description={the time point associated with the first stratum deposition}
}

\newglossaryentry{time}
{
    name=time,
    description={number of depositions elapsed between two time points}
}

\newglossaryentry{layer age}
{
    name=layer age,
    description={the number of depositions elapsed since a layer's deposition time}
}

\newglossaryentry{record depth}
{
    name=record depth,
    description={the number of depositions elapsed onto the historical record --- the number of layers within a historical record}
}

\newglossaryentry{layer}
{
    name=layer,
    description={position within a historical record associated with a single time point, which may or may be occupied}
}

\newglossaryentry{retained/pruned layer}
{
    name=retained/pruned layer,
    description={a layer within a historical record with present/removed strata, respectively}
}

\newglossaryentry{layer time point}
{
    name=layer time point,
    description={the deposition time associated with a layer}
}

\newglossaryentry{pruning}
{
    name=pruning,
    description={deletion of strata from a historical record (i.e., to reduce space occupied by the record) --- also used to refer to deletion of perfect tracking records for extinct lineages,}
}

\newglossaryentry{retention}
{
    name=retention,
    description={the act of carrying over a stratum into the next time point during the stratum deposition process}
}

\newglossaryentry{gap}
{
    name=gap,
    description={layers associated with contiguous time points that have been pruned ---- introduces inference uncertainty when comparing two columns}
}

\newglossaryentry{gap width}
{
    name=gap width,
    description={the number of contiguous time points that have been pruned --- gap with increases inference uncertainty}
}

\newglossaryentry{sparse/dense retention}
{
    name=sparse/dense retention,
    description={refers to relatively wide or relatively tight gap width, respectively}
}

\newglossaryentry{gap width distribution}
{
    name=gap width distribution,
    description={how gap widths relate to layer deposition times}
}

\newglossaryentry{resolution}
{
    name=resolution,
    description={the width of the gap containing a pruned layer or immediately following a retained layer (may be zero in the case of two successive retained strata)}
}

\newglossaryentry{stratum retention policy algorithm}
{
    name=stratum retention policy algorithm,
    description={the decision-making procedure of which strata to prune at each time point (also referred to simply as ``policy'')}
}

\newglossaryentry{policy resolution guarantee}
{
    name=policy resolution guarantee,
    description={upper bounds on resolutions across layers of a historical record with respect to layer age and/or record depth}
}

\newglossaryentry{extant record size}
{
    name=extant record size,
    description={the quantity of strata retained within a historical record at a particular time point}
}

\newglossaryentry{extant record order of growth}
{
    name=extant record order of growth,
    description={the asymptotic scaling relationship between the extant record size and record depth; see Section \ref{sec:extant_record_oog}}
}

\newglossaryentry{pruning enumeration}
{
    name=pruning enumeration,
    description={calculation of the set of strata to be pruned at a particular time point under a retention policy}
}

\newglossaryentry{policy enactment}
{
    name=policy enactment,
    description={the act of performing pruning enumeration and deleting strata with enumerated deposition times}
}

\newglossaryentry{update}
{
    name=update,
    description={the process of performing a deposition and applying policy enactment}
}

\newglossaryentry{update time complexity}
{
    name=update time complexity,
    description={the scaling relationship associated with the number of computational operations necessary to perform an update}
}

\newglossaryentry{founding stratum}
{
    name=founding stratum,
    description={the first stratum deposited into the historical record, the oldest stratum}
}

\newglossaryentry{newest stratum}
{
    name=newest stratum,
    description={the most recent stratum deposited into the historical record}
}

\newglossaryentry{extant record}
{
    name=extant record,
    description={the set of strata that have been retained through the policy at the present time point}
}

\newglossaryentry{extant record enumeration}
{
    name=extant record enumeration,
    description={calculation of deposition times present in the extant record at a time point under a retention policy}
}

\newglossaryentry{policy self-consistency}
{
    name=policy self-consistency,
    description={the requirement for each deposition time within an extant record enumeration to be consistently present in all extant enumerations since that deposition time}
}

\newglossaryentry{historical record}
{
    name=historical record,
    description={refers to the set of layers defined up to the current time point, also referred to as a ``record''}
}

\newglossaryentry{hereditary stratigraphic column}
{
    name=hereditary stratigraphic column,
    description={container for a historical record --- in phylogenetic applications, associated with a digital population member, also referred to as a ``column''}
}

\newglossaryentry{annotation}
{
    name=annotation,
    description={the one-to-one association of hereditary stratigraphic records with individual digital organisms to facilitate phylogenetic analysis}
}

\newglossaryentry{inheritance}
{
    name=inheritance,
    description={the act of copying the parent organism's hereditary stratigraphic column annotation and performing an update to create the offspring organism's hereditary stratigraphic column annotation during a reproduction event}
}

\newglossaryentry{inference}
{
    name=inference,
    description={best-effort estimation of historical phylogenetic relationships from extant hereditary stratigraphic columns}
}

\newglossaryentry{perfect tracking}
{
    name=perfect tracking,
    description={maintenance of an exact record of phylogenetic events during an evolutionary simulation}
}

\newglossaryentry{stream curation problem}
{
    name=stream curation problem,
    description={poses the question of how to satisfactorily maintain a temporally representative collection of stored observations on a rolling basis as new observations stream in}
}
