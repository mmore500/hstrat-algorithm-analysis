\section{Conclusion} \label{sec:conclusion}

In this work, we have characterized the stream curation problem --- how to maintain a temporally-representative running archive of stream data --- and provided several algorithms to solve it, each meeting different criteria.
We have systematized curatorial properties of data retention strategies, covering how many data items should be retained and how retained data items should distribute over past time.
We provide five policy algorithms that target a spectrum of size/coverage trade-offs and demonstrate procedures to enact them efficiently.
Implementations enable key optimizations to trim archive storage size overhead through efficient and computationally reducible (i.e., stateless) determination of retained contents.

Within the original context for their development, presented stream curation algorithms are key to tunability and efficiency of hereditary stratigraphy.
However, the stream curation problem generalizes beyond hereditary stratigraphy, and memory-smart optimizations provided here stand to boost the data stream mining capabilities of low-grade hardware in roles such as sensor nodes and data logging devices.

Much work remains.
We are particularly interested in exploring further adaptations to our approach to stream curation to improve efficacy of space use, efficiency of the update process, and simplicity of implementation for fixed-space contexts.
One promising direction involves adapting retention policies to directly specify a buffer index position to replace.
This would remove complications of shuffling down entries when early buffer entries are removed and, after initial population of the buffer, would ensure completely full space utilization.
Plenty remains for theoretical analysis, as well.
In particular, the extent to which stream curation policy algorithms minimize the number of records necessary to achieve their guarantees should be considered.

Ultimately, however, our interest in stream curation is application-driven.
To these ends, reference implementations of most algorithms described are provided in the public-facing \texttt{hstrat} Python library \citep{moreno2022hstrat}.
In addition to hereditary stratigraphy-specific tools, the \texttt{hstrat} library makes stream curation algorithm implementations available via a few lines of code.
