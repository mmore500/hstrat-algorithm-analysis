\section{Conclusion} \label{sec:conclusion}

We have documented the algorithmic backing of the emerging ``hereditary stratigraphy'' methodology for distributed lineage tracking of replicating digital artifacts.
This method works by annotating replicating entities with a heritable annotation, and then comparing end-state annotations after the fact to estimate their phylogenetic relationships --- akin to how biologists infer phylogenetic relationships from biological sequences.
We have shown the underlying mechanism of hereditary stratigraphy to be an instance of a more general ``streaming curation'' problem.
This problem requires online subsampling of a rolling observation stream to balance requirements to maintain information over stream history and to limit the number of retained observations.
In addition to hereditary stratigraphy, the streaming curation problem is faced in use cases involving memory-constrained data logger or wireless sensor network devices.
We identify five policy algorithms that provide different size/coverage trade-offs for the streaming curation problem, and provide procedures to enact them efficiently.

We then explore postprocessing algorithms used to compare and interpret hereditary stratigraphic annotations.
In particular, we sketch a trie-based tree reconstruction algorithm that proceeds from the underlying structure of the hereditary stratigraphy annotation.
As a point of comparison, we provide an algorithmic formalization of the established perfect-tracking algorithms currently standard in agent-based evolutionary modeling.
We find that both perfect tracking and hereditary stratigraphy have circumstances in which they excel.
Hereditary stratigraphy enables tracking in noisy, parallel, and distributed environments where perfect tracking would be impractical.
It also makes stronger guarantees about memory use (although on average perfect tracking uses less memory than most hereditary stratigraphy policies).
Lastly, it allows different tradeoffs between data resolution, accuracy, and memory use.
%TODO talk about the take away of comparing the two
% ELD: Took a stab at it

Much future work remains.
We are particularly interested in adapting hereditary stratigraphy to even more effectively make use of space in fixed-capacity contexts and minimize rearrangement of strata in the underlying annotation data structure (e.g., array of strata) during the update process.
One promising direction involves adapting retention policies to directly specify the index positions to replace rather than the deposit times of strata to prune.
Additionally, more general applications of the streaming curation problem would be worth exploring more directly.
There is also more theoretical analysis to be done.
Most notably, through the lens of information theory it would be possible to conduct a more formal analysis of the information-retention trade-offs among different policy algorithms. 
This work would also open the door to a more precise analysis of the relationship between pairwise relatedness estimation accuracy and overall tree reconstruction quality.
Finally, it would be worth assessing the extent to which curation policy algorithms minimize the number of records necessary to achieve their guarantees.

Typically, phylogeneticists tailor algorithms to the demands of available data, not the other way around.
This notion invites deeper consideration of phylogenetic information encoding within genomes.
For example, it would be interesting to explore upper bounds on phylogenetic reconstruction accuracy given genome length for different generational depths and population sizes.

Ultimately, our interest in hereditary stratigraphy and streaming curation is application-driven.
To these ends, reference implementatons of most algorithms described are provided in the public-facing \texttt{hstrat} and \texttt{phylotackpy} Python libraries \citep{moreno2022hstrat, dolson2023phylotrackpy}.
These libraries enable application of perfect phylogenetic tracking, hereditary stratigraphy, and streaming curation techniques in a few lines of code.
