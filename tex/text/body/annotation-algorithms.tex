\section{Streaming Curation Algorithms} \label{sec:annotation-algorithms}

% \pragmaonce

% adapted from https://www.overleaf.com/learn/latex/Commands
\providecommand{\dissertationelse}[2]{%
% adapted from https://tex.stackexchange.com/a/33577
\ifdefined\DISSERTATION
#1
\else
#2
\fi
}

% \pragmaonce

% adapted from https://www.overleaf.com/learn/latex/Commands
\providecommand{\dissertationonly}[1]{%
% adapted from https://tex.stackexchange.com/a/33577
\ifdefined\DISSERTATION%
#1%
\else%
\fi
}


\begin{figure*}
  \centering
  \footnotesize
  \begin{tabular}{m{0.07\textwidth}@{}|c@{}|c@{\hskip 0.01\textwidth}|m{0.14\textwidth}}
%-------------------------------------------------------------------------------
\hspace{-1ex}Policy&Lower-Resolution Parameterization&Higher-Resolution Parameterization&\makecell[c]{Properties}\\\hline
%-------------------------------------------------------------------------------
    \rotatebox{90}{\textbf{Fixed Resolution}}
  &
    \makecell{
      \includegraphics[valign=t,width=0.3\textwidth]{submodules/hereditary-stratigraph-concept-binder/binder/retention-policies/teeplots/fixed_resolution=512+num_layers=1024+stratum_retention_predicate=fixed-resolution+viz=tweaked-stratum-retention-drip-plot+ext=}
    }
  &
    \makecell{
      \includegraphics[valign=t,width=0.3\textwidth]{submodules/hereditary-stratigraph-concept-binder/binder/retention-policies/teeplots/fixed_resolution=128+num_layers=1024+stratum_retention_predicate=fixed-resolution+viz=tweaked-stratum-retention-drip-plot+ext=}
    }
  &
  \makecell[{{p{0.14\textwidth}}}]{
  \centering
    \bf{Space Complexity}\\
    $\mathcal{O}(n)$\\
    \bf{MRCA Uncertainty}\\
    $\mathcal{O}(1)$
  }
  \makecell[{{p{0.14\textwidth}}}]{
  \raggedright
    where $n$ is gens elapsed.
  }\\\hline
%-------------------------------------------------------------------------------
    \adjustbox{
      minipage=10em,
      rotate=90,
    }{
      \centering
      \textbf{Depth-Proportional\\Resolution}
      \par
    }
  &
    \makecell{
      \includegraphics[valign=t,width=0.3\textwidth]{submodules/hereditary-stratigraph-concept-binder/binder/retention-policies/teeplots/guaranteed_depth_proportional_resolution=1+num_layers=1024+stratum_retention_predicate=depth-proportional-resolution+viz=tweaked-stratum-retention-drip-plot+ext=}
    }
  &
    \makecell{
      \includegraphics[valign=t,width=0.3\textwidth]{submodules/hereditary-stratigraph-concept-binder/binder/retention-policies/teeplots/guaranteed_depth_proportional_resolution=4+num_layers=1024+stratum_retention_predicate=depth-proportional-resolution+viz=tweaked-stratum-retention-drip-plot+ext=}
    }
  &
  \makecell[{{p{0.14\textwidth}}}]{
  \centering
    \bf{Space Complexity}\\
    $\mathcal{O}(1)$\\
    \bf{MRCA Uncertainty}\\
    $\mathcal{O}(n)$
  }
  \makecell[{{p{0.14\textwidth}}}]{
  \raggedright
    where $n$ is gens elapsed.
  }\\\hline
%-------------------------------------------------------------------------------
  \adjustbox{
    minipage=10em,
    rotate=90,
  }{
    \centering
    \textbf{Recency-proportional\\Resolution}
    \par
  }
  &
  \makecell{
    \includegraphics[valign=t,width=0.3\textwidth]{submodules/hereditary-stratigraph-concept-binder/binder/retention-policies/teeplots/guaranteed_mrca_recency_proportional_resolution=0+num_layers=1024+stratum_retention_predicate=recency-proportional-resolution+viz=tweaked-stratum-retention-drip-plot+ext=}
  }
  &
  \makecell{
    \includegraphics[valign=t,width=0.3\textwidth]{submodules/hereditary-stratigraph-concept-binder/binder/retention-policies/teeplots/guaranteed_mrca_recency_proportional_resolution=4+num_layers=1024+stratum_retention_predicate=recency-proportional-resolution+viz=tweaked-stratum-retention-drip-plot+ext=}
  }
  &
  \makecell[{{p{0.14\textwidth}}}]{
  \centering
    \bf{Space Complexity}\\
    $\mathcal{O}(\log(n))$\\
    \bf{MRCA Uncertainty}\\
    $\mathcal{O}(m)$
  }
  \makecell[{{p{0.14\textwidth}}}]{
  \raggedright
    where $m$ is gens since MRCA and $n$ is total gens elapsed.
  }\\\hline
%-------------------------------------------------------------------------------
    \adjustbox{
      minipage=10em,
      rotate=90,
    }{
      \centering
      \textbf{Geometric Sequence\\$n$th Root}
      \par
    }
  &
    \makecell{
      \includegraphics[valign=t,width=0.3\textwidth]{submodules/hereditary-stratigraph-concept-binder/binder/retention-policies/teeplots/guaranteed_depth_proportional_resolution=1+num_layers=1024+stratum_retention_predicate=tapered-depth-proportional-resolution+viz=tweaked-stratum-retention-drip-plot+ext=}
    }
  &
    \makecell{
      \includegraphics[valign=t,width=0.3\textwidth]{submodules/hereditary-stratigraph-concept-binder/binder/retention-policies/teeplots/guaranteed_depth_proportional_resolution=4+num_layers=1024+stratum_retention_predicate=tapered-depth-proportional-resolution+viz=tweaked-stratum-retention-drip-plot+ext=}
    }
  &
  \makecell[{{p{0.14\textwidth}}}]{
  \centering
    \bf{Space Complexity}\\
    $\mathcal{O}(1)$\\
    \bf{MRCA Uncertainty}\\
    $\mathcal{O}(n)$
  }
  \makecell[{{p{0.14\textwidth}}}]{
  \raggedright
    where $n$ is gens elapsed.
  }\\\hline
%-------------------------------------------------------------------------------
    \adjustbox{
      minipage=10em,
    rotate=90,
    }{
      \centering
      \textbf{Curbed Recency-proportional Resolution}
      \par
    }
  &
    %TODO generate this graphic
    \makecell{
      \includegraphics[valign=t,width=0.3\textwidth]{submodules/hereditary-stratigraph-concept-binder/binder/retention-policies/teeplots/guaranteed_depth_proportional_resolution=1+num_layers=1024+stratum_retention_predicate=tapered-depth-proportional-resolution+viz=tweaked-stratum-retention-drip-plot+ext=}
    }
  &
  %TODO generate this graphic
    \makecell{
      \includegraphics[valign=t,width=0.3\textwidth]{submodules/hereditary-stratigraph-concept-binder/binder/retention-policies/teeplots/guaranteed_depth_proportional_resolution=4+num_layers=1024+stratum_retention_predicate=tapered-depth-proportional-resolution+viz=tweaked-stratum-retention-drip-plot+ext=}
    }
  &
  \makecell[{{p{0.14\textwidth}}}]{
  \centering
    \bf{Space Complexity}\\
    $\mathcal{O}(1)$\\
    \bf{MRCA Uncertainty}\\
    $\mathcal{O}(n)$
  }
  \makecell[{{p{0.14\textwidth}}}]{
  \raggedright
    where $n$ is gens elapsed.
  }


  \end{tabular}
  \caption{
  Comparison of stratum retention policies.
  Policy visualizations show retained strata in black.
  Time progresses along the $y$-axis from top to bottom.
  New strata are introduced along the diagonal and then ``drip'' downward as a vertical line until eliminated.
  The set of retained strata present within a column at a particular generation $g$ can be read as intersections of retained vertical lines with a horizontal line with intercept $g$.
  Policy visualizations are provided for two parameterizations for each policy: the first where the maximum uncertainty of MRCA generation estimates would be 512 generations and the second where the maximum uncertainty of MRCA generation estimates would be 128 generations.
  }
  \label{fig:retention-policies}
\end{figure*}


This section collects specification and validation of five proposed streaming curation algorithms.
These algorithms operate online on a rolling stream of incoming observations to maintain a representative subsample of retained observations.
Streaming curation algorithms differ in the growth rate allowed for the curated observation collection and with respect to relative prioritization of retaining recent observations compared to older observations.

% For each surveyed algorithm, we present
% \begin{itemize}
% \item narration of retention policy algorithm strategy,
% \item retention policy algorithm extant record size order of growth (i.e., the scaling relationship between record depth and curated collection size),
% \item resolution guarantees (i.e., bounds on gap size between retained observations),
% \item instructions to enumerate time points retained within the curated subsample for any record depth,
% \item validation of policy algorithm self-consistency (i.e., that preceding record depths retain all observation time points required at later record depths),
% \item an algorithm for policy enactment (i.e., what observations to drop at each successive depth), and
% \item reflection of likely use case scenarios.
% \end{itemize}

We introduce the following five curation policy algorithms,
\begin{itemize}
\item Fixed-Resolution (FR) Policy Algorithm (Section \ref{sec:fixed-resolution-algo}),
\item Depth-Proportional Resolution (DPR) Policy Algorithm (Section \ref{sec:depth-proportional-resolution-algo}),
\item Recency-Proportional Resolution (RPR) Policy Algorithm (Section \ref{sec:recency-proportional-resolution-algo}),
\item Geometric Sequence $n$th Root(GSNR) Policy Algorithm (Section \ref{sec:geom-seq-nth-root-algo}), and
\item Curbed Recency-Proportional Resolution (CRPR) Policy Algorithm (Section \ref{sec:curbed-recency-proportional-resolution-algo}).
\end{itemize}
The accomanying \texttt{hstrat} library provides reference implementations for all five policy algorithms \citep{moreno2022hstrat}.

Figure \ref{fig:retention-policies} compares retention patterns induced by each algorithm and recaps each policy algorithm's principal properties.
FR and DPR follow even curation prioritization while RPR, GSNR, and CRPR follow recency-proportional curation prioritization.
Recency-proportional techniques are of particular interest due to predictions from coalescence theory \citep{nordborgCoalescentTheory2019, berestyckiRecentProgressCoalescent2009}, which suggest that in most cases there will be more lineage divergences in recent time points.
Thus, prioritizing recent data should often improve the accuracy of subsequent data analysis. 
% should we justify recency-proporitonal here?
% ELD: Added it
Collection size grows the most aggressively under FR as $\mathcal{O}(n)$.
RPR reduces collection size growth to $\mathcal{O}(\log n)$.
The remaining algorithms enforce a fixed cap on curated collection size.
Note that GSNR and CRPR exhibit identical asymptotic properties.
We include both, as CRPR is an engineered extension of GSNR that improves the efficacy of available space usage during initial shallow record depth.

Appropriate algorithm choice will depend on use case scenario.
Relevant criteria to consider include
\begin{itemize}
  \item uncertainty and magnitude of upper bounds on record depth, if any,
  \item available storage capacity,
  \item relative importance of recent and ancient observations, and
  \item any hard record quality requirements (i.e., maximum acceptable gap size).
\end{itemize}

In a real-time scenario, record depth bounds would be considered in terms of upper bounds on chronological duration of record collection and the real-time sampling rate of observations.

\subsection{Fixed Resolution (FR) Policy Algorithm}
\label{sec:fixed-resolution-algo}

\begin{algorithm}
\caption{Fixed Resolution Stratum Discard Generator}
\label{alg:fixed-resolution-gen-drop-ranks}
\begin{algorithmic}[1]
    \STATE{ $\text{spec} \gets \text{policy.GetSpec()}$ }
    \FOR{$i = 0$ \textbf{to} $\texttt{num\_\text{strata}\_\text{deposited}} - 1$ \textbf{step} $\text{spec.GetFixedResolution()}$}
        \STATE{ \textbf{yield} $i$ }
    \ENDFOR
    \STATE{ $last\_rank \gets \texttt{num\_\text{strata}\_\text{deposited}} - 1$ }
    \IF{$last\_rank > 0$ \textbf{and} $last\_rank \mod \text{spec.GetFixedResolution()} \neq 0$}
        \STATE{ \textbf{yield} $last\_rank$ }
    \ENDIF
\end{algorithmic}
\end{algorithm}

\begin{algorithm}
\caption{Fixed Resolution Stratum Enumeration}
\label{alg:fixed-resolution-enum-retained-rans}
\begin{algorithmic}[1]
\STATE{Define $P:=T:=\{ \{1\},\ldots,\{d\}$\}}
\WHILE{$\#P > 1$}
\STATE\label{line3}{Choose $C^\prime\in\mathcal{C}_p(P)$ with $C^\prime :=
\operatorname{argmin}_{C\in\mathcal{C}_p(P)} \varrho(C)$}
\STATE{Find an optimal partition tree $T_{C^\prime}$ }
\STATE{Update $P := (P{\setminus} C^\prime) \cup \{ \bigcup_{t\in C^\prime} t \}$}
\STATE{Update $T := T \cup \{ \bigcup_{t\in\tau} t : \tau\in T_{C^\prime}{\setminus}
\mathcal{L}(T_{C^\prime})\}$}
\ENDWHILE
\RETURN $T$
\end{algorithmic}
\end{algorithm}


\begin{theorem}{Fixed Resolution Space Complexity}
\label{thm:fixed-resolution-algo-space-complexity}

\end{theorem}

\begin{proof}
\label{prf:fixed-resolution-algo-space-complexity}

\end{proof}

% TODO: missing proof
\begin{theorem}{Fixed Resolution Uncertainty Bound}
\label{thm:fixed-resolution-algo-uncertainty-bound}

\end{theorem}

\begin{proof}
\label{prf:fixed-resolution-algo-uncertainty-bound}

\end{proof}



\subsection{Depth-Proportional Resolution (DPR) Policy Algorithm}
\label{sec:depth-proportional-resolution-algo}

The depth-proportional resolution policy algorithm provides capped extant record size with even coverage over record history.
This guarantee requires retained observations to be spaced with a gap width proportional to record depth.
%An equivalent interpretation of DPR strategy would be interspersal of the historical record with a fixed number of waypoints.
Alternatively, DPR can be seen as interspersing the historical record with a fixed number of waypoints.

The set of items retained under depth-proportional resolution is defined as,
\begin{align}
\mathsf{DPR\_kept}(\colorT)
&= \{
\colorTbar \in [0\twodots\colorT)
: \colorTbar \bmod R = 0 \text{ or } \colorTbar = \colorT - 1
\}
\text{ where }
R = \max(\left \lfloor \colorT / r \right \rfloor_{\mathrm{bin}}, \;\; 1).
\label{eqn:dpr_kept}
\end{align}
We demonstrate self-consistency over elapsed time $\colorT$ in Suplementary Theorem \ref{thm:depth-proportional-resolution-algo-self-consistency}.

Because observation time points are immutable after the fact, translating this naive DPR plan to a rolling, ``online'' basis necessitates a further consideration.
To conservatively maintain resolution guarantees, it is acceptable to err on the side of caution by choosing gap sizes smaller than the worst-case requirement.
This approach allows a simple trick for achieving policy self-consistency: flooring gap sizes to the next lower power of two.
Under this scheme, gap size will periodically double.
Beacause multiples of a binary power superset multiples of higher binary power, self-consistency is maintained.
As intuition, therefore, the full DPR policy algorithm can be conceived of through a simple principle: %successive eliminations of every other retained observation when a capacity threshold is reached.
each time a capacity threshold is reached, every second observation is eliminated.

% Binary floor approximation exhibits another useful property: policy algorithm property distortions can be bounded by a constant.
% Minimum gap width is half of the highest allowed gap width, while maximum extant record count is twice the lowest allowed count.
Policy algorithm behavior is parametrized by a minimum number of bin windows over record history, $r$.
All gap sizes are equal (or halved), so absolute resolution guarantee of at least $n/r$, with $n$ as record depth, applies.
Further, because binary flooring operations at most halve gap widths, record count at most doubles.
This property gives the record size cap of $2r$.

Algorithm \ref{alg:depth-proportional-resolution-algo-enum-retained-ranks} provides enumeration of retained time points under the DPR scheme.
Although this process can be achieved via set subtraction between enumerations at successive time points with $\mathcal{O}(1)$ complexity, Algorithm \ref{alg:depth-proportional-resolution-algo-gen-drop-ranks} provides a more expedient approach.
% Theorems \ref{thm:depth-proportional-resolution-algo-uncertainty-bound} and \ref{thm:depth-proportional-resolution-algo-space-complexity} cover extant record size order of growth and resolution guarantorship, respectively.
Figure \ref{fig:retention-policies} includes a time-lapse of the extant record under the DPR policy algorithm.

For simplicity, we have presented a bare-bones approach to depth-proportional resolution, where the entire record is simultaneously decimated by a factor of two upon reaching capacity.
This procedure results in regular episodes where extant record count instantaneously halves.
Such fluctuation may be undesirable.
Many use-cases for constant space complexity will arise from fixed memory allocation.
Such reserved memory cannot typically be used for other purposes, so any unused space would be wasted.
%Occupying only half of bounded capacity clearly comes out on the ``use it or lose it'' downside.

An alternate ``tapered'' variant of the depth-proportional resolution algorithm remedies this space-usage quirk.
%Instead of all at once,
The tapered approach eliminates phased-out observations one by one as new observations accrue, but otherwise has the same properties as the algorithms described for DPR.
% Supplementary Section \ref{sec:depth-proportional-resolution-tapered-algo} details this tapered DPR policy algorithm.
% TODO: ADD SECTION ^^^
The accompanying \texttt{hstrat} software library implements both variants.

% \begin{algorithm}
\caption{Depth-proportional Resolution Stratum Retention Predicate}
\label{alg:depth-proportional-resolution-algo-pred-keep-rank}
\begin{algorithmic}[1]
\STATE{Define $P:=T:=\{ \{1\},\ldots,\{d\}$\}}
\WHILE{$\#P > 1$}
\STATE{Find an optimal partition tree $T_{C^\prime}$ }
\STATE{Update $P := (P{\setminus} C^\prime) \cup \{ \bigcup_{t\in C^\prime} t \}$}
\STATE{Update $T := T \cup \{ \bigcup_{t\in\tau} t : \tau\in T_{C^\prime}{\setminus}
\mathcal{L}(T_{C^\prime})\}$}
\ENDWHILE
\RETURN $T$
\end{algorithmic}
\end{algorithm}
.
\begin{algorithm}
\caption{Depth-proportional Resolution Stratum Enumeration}
\label{alg:depth-proportional-resolution-algo-enum-retained-ranks}

    \begin{algorithmic}
        \Require{ $\texttt{n}$ -- the number of strata deposited }
        \Require{ $\texttt{r}$ -- the fixed resolution desired }
        \Ensure{ array of retained strata }
    \end{algorithmic}


    \begin{algorithmic}[1]
        \State $\texttt{uncertainty} \gets (\text{largest integral power of two} \le x) \lor 1$
        \State $\texttt{arr} \gets \text{empty array of length } \frac{n}{\texttt{uncertainty}} + 1$

        \FOR{$i = 0$ \textbf{to} $\texttt{n} - 1$ \textbf{step} $\text{uncertainty}$}
            \State $\texttt{arr} [$i$] \gets i$
        \EndFor
        \State $\texttt{last\_rank} \gets \texttt{n} - 1$
        \If{$\texttt{last\_rank} > 0$ \textbf{and} $\texttt{last\_rank} \mod \texttt{uncertainty} \neq 0$}
            \State $\texttt{\texttt{last\_rank}} [$i$] \gets \texttt{last\_rank}$
        \EndIf
    \end{algorithmic}
\end{algorithm}

% TODO: missing pseudocode
\begin{algorithm}
\caption{Depth-proportional Resolution Discard Generator}
\label{alg:depth-proportional-resolution-algo-gen-drop-ranks}
\begin{algorithmic}[1]
\STATE{Define $P:=T:=\{ \{1\},\ldots,\{d\}$\}}
\WHILE{$\#P > 1$}
\STATE{Find an optimal partition tree $T_{C^\prime}$ }
\STATE{Update $P := (P{\setminus} C^\prime) \cup \{ \bigcup_{t\in C^\prime} t \}$}
\STATE{Update $T := T \cup \{ \bigcup_{t\in\tau} t : \tau\in T_{C^\prime}{\setminus}
\mathcal{L}(T_{C^\prime})\}$}
\ENDWHILE
\RETURN $T$
\end{algorithmic}
\end{algorithm}

% \begin{theorem}{Depth-proportional Resolution Space Complexity}
\label{thm:depth-proportional-resolution-algo-space-complexity}

\end{theorem}

\begin{proof}
\label{prf:depth-proportional-resolution-algo-space-complexity}
\end{proof}

% \begin{theorem}{Depth-proportional Resolution Uncertainty Bound}
\label{thm:depth-proportional-resolution-algo-uncertainty-bound}

\end{theorem}

\begin{proof}
\label{prf:depth-proportional-resolution-algo-uncertainty-bound}

\end{proof}



\subsection{Recency-proportional Resolution (RPR) Policy Algorithm}
\label{sec:recency-proportional-resolution-algo}

This streaming curation algorithm's properties fall between the properties of the fixed resolution (FR) and depth-proportional resolution (DPR) policy algorithms, covered in the immediately preceding sections.

Recall that the DPR policy algorithm's gap widths grow in linear proportion to record depth.
In contrast, the fixed resolution algorithm's gap width remains constant below a specified bound across record depths.

The recency-proportional resolution (RPR) policy algorithm bounds gap width to a linear factor of layer age (i.e., time steps back from the newest layer).
Suppose $n$ observations have elapsed.
% set k as age and explain that then replace n - m
% annotate-equations (latex math annotation?)
Then, for a user-specified constant $r$, no gap width for layer $m$ will exceed size
\begin{align}
  \left\lfloor \frac{n - m}{r} \right\rfloor.
  \label{eqn:rpr-gap}
\end{align}
Resolution at each layer widens linearly with record depth.
Consequently, resolution widens in linear proportion to layer age.
Resolution for any given layer age, however, remains constant for all record depths.

The FR and DPR policy algorithms exhibit $O(r)$ and $O(rn)$ extant record orders of growth, respectively.
We will show extant record order of growth as $O(r\log{n})$ under the recency-proportional resolution policy algorithm.

Algorithm \ref{alg:recency-proportional-algo-gen-drop-ranks} enumerates time points of dropped observations under the RPR policy algorithm.
Figure \ref{fig:retention-policies} includes a time-lapse of the extant record under this policy algorithm.

The extant record is determined iteratively, beginning at observation time zero --- which is always retained.
Per Equation \ref{eqn:rpr-gap}, gap width to the next retained observation can be at most $\lfloor n/r \rfloor$ sites, where $n$ is record depth.
Although selecting to retain the $\lfloor n/r \rfloor$ observation time observation would satisfy policy resolution guarantees, a slight complication is necessary to ensure that retained observations have previously been deleted.
A fuller self-consistency rationale will follow, but in short gap width is floored to the next lower power of two,
\begin{align*}
  2^{\lfloor \log_{2}\left(\frac{n}{r}\right) \rfloor}.
\end{align*}
The next iteration repeats the procedure from the newly retained observation time instead of observation time zero.
Iteration continues until reaching the newest observation.

The set of observations to eliminate can be calculated from set subtraction between enumerations of the historical record at time points $t-1$ and $t$.
So, update time complexity follows from extant record enumeration time complexity, which turns out to be $O(\log n)$ as detailed in Theorem \ref{thm:recency-proportional-resolution-algo-uncertainty-bound}.
We provide a tested, but unproven, constant-time pruning enumeration implementation in the \texttt{hstrat} library, but will not cover it here. % mention why?
The extant record order of growth $O(\log n)$ also follows closely from the record enumeration algorithm, as detailed in Theorem \ref{thm:recency-proportional-resolution-algo-space-complexity}.

% TODO figure?

Why does flooring step sizes to a binary power ensure self-consistency?
Let us begin by noting properties applicable to all layers $l$,
\begin{enumerate}
\item gap width provided at retained layer $l$ increases monotonically as record depth grows,
\item the retained observation preceding or at $l$ has observation time at an even multiple of surrounding gap widths, and
\item all observations at time points that are multiples of gap width past $l$ up to the newest observation are retained.
\end{enumerate}
Observe that gap width decreases monotonically with decreasing layer age (i.e., increasing layer recency).

Properties 2 and 3 occur as a result of stacking monotonically-decreasing powers of two.
Subsequent smaller powers of two tile evenly to all multiples of a larger power of 2, giving property 3.
Conversely, preceding larger powers of 2 can be evenly divided by succeeding smaller powers of 2, ensuring that the edges of smaller powers of 2 gaps occur at even multiples of their gap width, giving property 2.

Under the binary flooring procedure, when gap size increases at a layer it will double (or quadruple, octuple, etc.).
Availability of the new gap endpoint after a gap size increase is guaranteed from the tiling properties due to that endpoint being an even multiple of original step size.
Theorem \ref{thm:recency-proportional-resolution-algo-self-consistency} uses inductive proof-by-contradiction to show this self-consistency.

The RPR policy algorithm provides stable relative accuracy indefinitely.
This makes it particularly attractive in phylogenetic tracking scenarios using hereditary stratigraphy where guarantees about branch length accuracy are critical.
At comparable annotation sizes, we have found that recency-proportional distribution of gap widths outperforms even gap width distribution in phylogenetic information recovery \citep{moreno2022hereditary}.
Preliminarily, maintaining 3\% relative precision appears sufficient to eliminate most bias from reconstruction error on phylogenetic metrics \citep{moreno2023toward}.
%TODO add details about coalescent theory being a good idea for recency-proportional

The RPR policy algorithm's indefinite stability may be particularly useful in scenarios of indefinite or indeterminate record keeping duration.
Although annotation extant record size grows unboundedly, logarithmic memory usage growth is manageable in most practical scenarios.
However, this policy would not suit applications with hard caps on annotation size (e.g., fixed memory footprint digital genomes).

\begin{algorithm}
\caption{Recency-proportional Resolution Stratum Discard Generator}
\label{alg:recency-proportional-algo-gen-drop-ranks}
\begin{algorithmic}
    \Require{ $\texttt{n}$ -- the number of strata deposited }
    \Require{ $\texttt{r}$ -- the fixed resolution desired }
    \Ensure{ array of dropped strata }

    \Procedure{NumberToCondemn}{$\texttt{n}, \texttt{r}$}
        \If{$(\texttt{n} \bmod 2 = 1) \lor (\texttt{n} < 2 \cdot \texttt{r} + 1)$}
            \Return $0$
        \Else
            \Return $1 + \Call{NumberToCondemn}{$\texttt{n} / 2$, \texttt{r}}$
        \EndIf
    \EndProcedure

    \State $\texttt{num\_to\_condemn} \gets \Call{NumberToCondemn}{\texttt{n}, \texttt{r}}$
    \State $\texttt{arr} \gets \text{empty array of length num\_to\_condemn}$

    \For{$i = 0$ \textbf{to} $\texttt{num\_to\_condemn} - 1$}
        \State $\texttt{arr} [$i$] \gets \texttt{n} - 2^{i} \cdot (2 \texttt{r} + 1)$
    \EndFor
\end{algorithmic}
\end{algorithm}

% \begin{algorithm}
\caption{Recency-proportional Stratum Retention Predicate}
\label{alg:recency-proportional-resolution-algo-pred-keep-rank}
\begin{algorithmic}[1]
    \STATE $\text{resolution} \gets \text{spec.GetRecencyProportionalResolution()}$
    \STATE $\text{cur\_rank} \gets 0$
    \STATE $last\_rank \gets \texttt{num\_\text{strata}\_\text{deposited}} - 1$
    \WHILE{$last\_rank - \text{cur\_rank} > \text{resolution}$}
        \STATE \textbf{yield} $\text{cur\_rank}$
        \STATE $\text{cur\_rank} \gets \text{cur\_rank} + \texttt{calc\_provided\_uncertainty(} \text{resolution}, \text{last\_rank - cur\_rank} \texttt{)}$
    \ENDWHILE
    \FOR{$i = \max(\text{last\_rank} - \text{resolution}, 0)$ \textbf{to} $\texttt{num\_\text{strata}\_\text{deposited}} - 1$}
        \STATE \textbf{yield} $i$
    \ENDFOR
\end{algorithmic}
\end{algorithm}

% \begin{algorithm}
\caption{Recency Proportional Resolution Stratum Enumeration}
\label{alg:recency-proportional-resolution-algo-enum-retained-ranks}
\begin{algorithmic}[1]
    \Require{ \colorT -- the number of strata deposited }
    \State $\colorTbar \gets 0$
    \While{$\colorTbar < \colorT$}
        \State \textbf{yield} $\colorTbar$
        \State $\colorTbar \gets \colorTbar + \max\Big(\left \lfloor \frac{\colorT  - 1 - \colorTbar}{r + 1} \right \rfloor_{\mathrm{bin}}, \;\; 1\Big)$
    \EndWhile
\end{algorithmic}
\end{algorithm}


% \begin{theorem}{Recency-proportional Algorithm Self-Consistency}
\label{thm:recency-proportional-algo-self-consistency}
Let $\mathsf{RPR\_kept}(n)$ be the set of data points retained at time point $n$, defined per Equation \ref{eqn:dpr_kept}.
Showing self-consistency requires that we demonstrate $\forall m \geq 0, n \in [0 \twodots m)$,
\begin{align*}
\mathsf{RPR\_kept}(n)
\geq
\mathsf{RPR\_kept}(m)
\setminus
\mathsf{future}(n, m).
\end{align*}
where $\mathsf{future}(n, m) = [n \twodots m - 1)$.
\end{theorem}

\begin{proof}
\label{prf:recency-proportional-algo-self-consistency}
Our goal is equivalent to showing that $\forall m \geq 0, n \in [0 \twodots m)$,
\begin{align*}
\mathsf{RPR\_kept}(n) \cup \mathsf{future}(n, m) \geq \mathsf{RPR\_kept}(m).
\end{align*}

Let
\begin{align*}
R_m
&=
\max\Big(\left \lfloor \frac{m  - 1 - \colorTbar}{r + 1} \right \rfloor_{\mathrm{bin}}, \;\; 1\Big)\\
R_n
&=
\max\Big(\left \lfloor \frac{n  - 1 - \colorTbar}{r + 1} \right \rfloor_{\mathrm{bin}}, \;\; 1\Big).
\end{align*}
Note that we have both $R_m, R_n \in 2^{\mathbb{N}}$.
Further, because $n < m$ we have $R_n \leq R_m$.
From these two observations, we may remark that $R_n$ evenly divides $R_m$
\begin{align}
R_m \bmod R_n = 0.
\label{eqn:rmrn0a}
\end{align}

Suppose $\colorTbar \in \mathsf{RPR\_kept}(m)$.
In this case, at least one of the following is true (A) $k = m - 1$ or (B) $k \bmod R_m = 0$.

In the first case, $\colorTbar \stackrel{\checkmark}{\in} \mathsf{future}(n, m)$.
In the second case, we may conclude from Equation \ref{eqn:rmrn0a} that $\colorTbar \bmod R_n = 0$.
So, $\colorTbar \stackrel{\checkmark}{\in} \mathsf{RPR\_kept}(n)$.
\end{proof}

\begin{theorem}{Recency-proportional Resolution Space Complexity}
\label{thm:recency-proportional-resolution-algo-space-complexity}

The \gls{extant record size} of the Recency-proportional Resolution Policy Algorithm grows with order $\mathcal{theta}{(k \log{n})}.$

\end{theorem}

\begin{proof}
\label{prf:recency-proportional-resolution-algo-space-complexity}
As per \ref{sec:extant_record_oog}, we will set out to prove that output array of this policy algorithm has an order of growth of $\mathcal{theta}{(k \log{n})},$ where $k$ is a user-provided resolution and $n$ is the number of depositions.

Algorithm \ref{alg:recency-proportional-algo-gen-drop-ranks} determines the array of strata to be dropped at any given time point.
Observe that whenever $2 \mid n$ at least one stratum will be dropped.
More generally, for any positive integer $i \le log_2{n}$, we have that if $2^i \mid n$ then $i$ strata will be dropped. 
Thus, the number of dropped strata is bounded above by $\sum_{i=1}^{log_2{n}} n = n \log_2{n}.$
As such, the number of retained strata is bound by $n - n \log_2{n} \le \log_2{n}$ for all positive $n.$
Given that no strata are dropped when $\frac{n}{2} - 1 < k,$ we observe that the output array of this policy algorithm is bound above by $\mathcal{O}{(k \log{n})}.$
Via \ref{sec:extant_record_oog}, we can conclude that this bound is actually $\mathcal{theta}{(k \log{n})}.$
\end{proof}

% % TODO: missing proof
\begin{theorem}{Recency-proportional Resolution Uncertainty Bound}
\label{thm:recency-proportional-resolution-algo-uncertainty-bound}

\end{theorem}

\begin{proof}
\label{prf:recency-proportional-resolution-algo-uncertainty-bound}

\end{proof}



\subsection{Geometric Sequence $n$th Root (GSNR) Policy Algorithm}
\label{sec:geom-seq-nth-root-algo}

% The geometric sequence $n$th root (GSNR) policy algorithm tackles recency-proportional uncertainty objectives within a fixed memory capacity constraint.  is that ok??????? yeah that looks good!! fixed-size memory? among observations? i like it!
The geometric sequence $n$th root (GSNR) policy algorithm arranges recency-proportional gap sizes among a capped-size set of retained observations.
% Although recency-proportional gap size will not be bounded below a fixed threshold in this context, GSNR seeks to minimize it to the extent possible.
Although recency-proportional gap size will not be bounded to a fixed threshold in this context, GSNR seeks to minimize worst relative gap size as much as possible.

Recall that the recency-proportional policy algorithm (RPR) exhibits logarithmic growth in extant record size with respect to record depth $n$.
When an increased order of magnitude depth is reached, additional observations must be retained under the RPR algorithm.
For $a = \log_b(n)$, $a$ is proportional to extant record size.
Equivalently, $b^a = n$.
Under RPR, growth in extant record size can be roughly conceptualized as related to insufficiency of the base $b$ to reach $n$ within $a$ multiplicative steps.
Growth in $a$ --- i.e., additional multiplication by $b$ --- can be thought of in terms of adding a level of structural hierarchy within the layout of retained observations.
As $n$ increases, additional levels of structural hierarchy become necessary.
These additional hierarchical levels increase extant record size.
% Put another way, progression of the policy algorithm accommodates additional appending an additional level of hierarchy.
% ELD: I feel like I was supposed to come away from this paragraph understanding what problem GSNR is trying to solve, but I don't
% @ELD 2024: this comment applies more to the paragraph before this one now, but I think we're still not doing a great job of motivating GSNR

In order to prevent such unbounded growth, the GSNR policy algorithm fixes the number of hierarchical levels $a$ and accommodates additional record depth by adjusting the multiplicative factor $b$.
This scheme can be imagined enforcing as arrangement of $a$ exponentially-spaced target points along the historical record.
% ELD: I know we don't have time or space, but I think a little picture of this would help a lot
As time elapses, the quantity of target points remains constant.
The target points shift to fill $n$ by increasing their exponential spacing factor $b$.
The necessary magnitude of $b$ works out as $b = n^{1/a}$.
Target ages therefore correspond to $n^{0/a}, n^{1/a}, \ldots, n^{a/a}$.
This geometrically-spaced target point sequence eponymizes the GSNR policy algorithm.
Converting observation age to absolute time point, targets span $n - n^{0/a}, n - n^{1/a}, \ldots, n - n^{a/a}$.

% ELD: It might be worth trying to sign-post earlier that this algorithm is the way it is because that happens to work well, and that its not something that should be obvious to the audience

Now, attention turns to exploiting the $n$th root geometric targets to define a retention policy.
We will break the problem down to consideration of one individual target point $n - n^{x/a}$.
Under the constraint of $\mathcal{O}(1)$ total space for curated observations, we can only curate a fixed number of observations per target point.
We will seek to curate a fixed size collection of retained observations to bound gap size past the target point below $n^{x/a}$.

By nature of definition, target point times advance monotonically.
As a consequence, a retained observation can remain behind a target point indefinitely.
We will incorporate such coverage into our design --- let's call such a point behind the target the ``backstop'' $\beta$.
% Because the target point shifts with record depth, but retained time points are immutable, no retained backstop time point can exactly track the target point.
% ELD: Maybe "use" instead of "reach for"? Also maybe power of 2 instead of binary power? It took me a while to parse what a binary power is
% Instead, we will follow the vein of the RPR policy algorithm and the depth-proportional resolution policy algorithm and reach for a binary power.

We will use a power of 2 trick to maintain backstop coverage.
To begin, let us take the binary floor of half $n^{x/a}$,
% ELD: I'd either make this sentence have more intuition or cut it.
% TODO we should probably define binary floor in preliminaries and refer there
% addendum: we no longer have preliminaries
\begin{align*}
  \kappa(n)
  &=
  2^{\lfloor \log_{2}(n^{x/a}/2) \rfloor}.
\end{align*}
We will retain recent time points that are multiples of this value $\kappa$.
Note that $n$ strictly increasing implies $\kappa(n)$ monotonically increasing.

Let's define a floor $B$ to help place our backstop $\beta$,
\begin{align*}
  B(n)
  &=
  \max \left(
    n - \left\lceil  \frac{3n^{x/a}}{2} \right\rceil,
    0
  \right)
\end{align*}
By design, $B$ precedes target point $n - n^{x/a}$.
Again, with $x < a$, $n$ strictly increasing implies $B(n)$ monotonically increasing.

Rounding $B$ up to the next time point aligned to cadence $\kappa$ gives our backstop time point $\beta$,
\begin{align*}
  \beta(n)
  &=
  B(n) + \big(-B(n) \bmod \kappa(n)\big).
\end{align*}
% ELD: there's a weird gap in this equation, but it's probably not worth worrying about
% MAM: latex mod formatting is always weird and ugly
% SRP: \bmod is quirky and cute :3
It can be shown that $\beta(n) \leq n^{x/a}$.
Because $B(n)$ is monotonically increasing, $\beta(n)$ is as well.

We will retain the time point set $S_x$ comprising multiples of $\kappa$ at or after the backstop $\beta$,
\begin{align*}
  S_x(n) = \{\, t \mid \beta(n) \leq t < n \text{ and } t \mod \kappa(n) = 0 \,\}.
\end{align*}

Why are time points $S_x(n)$ guaranteed to be a subset of $S_x(n-1) \cup \{n\}$ (i.e., self consistency)?
Consider several non-mutually exclusive possible scenarios that could occur when transitioning from $n - 1$ to $n$,
\begin{enumerate}
  \item $\kappa$ changes: $\kappa(n - 1) \neq \kappa(n)$.

  Because $\kappa$ is monotonically increasing, $\kappa(n - 1) < \kappa(n)$.
  Binary flooring procedures have ensured $\kappa(n - 1)$ and $\kappa(n)$ are perfect powers of 2.
  Thus, $\kappa(n)$ is an even multiple of $\kappa(n - 1)$.
  So,
  \begin{align*}
    &\{\, t \mid \beta(n) \leq t < n \text{ and } t \mod \kappa(n) = 0 \,\}\\
    &\subseteq \{\, t \mid \beta(n) \leq t < n \text{ and } t \mod \kappa(n - 1) = 0 \,\}.
  \end{align*}
  Also, because $\beta$ monotonically increasing, $\beta{n - 1} \leq \beta{n}$.
  
  In conjunction, these stipulations give us $S_x(n) \subseteq \{S_x(n - 1), n\}$.
  % ELD: indentation is kind of wonky here

  \item $\beta$ changes: $\beta(n - 1) \neq \beta(n)$.

  Because $\beta$ is monotonically increasing, $\beta(n - 1) < \beta(n)$.
  We have $\beta(n) \mod \kappa(n) = 0$.
  Because $\kappa(n)$ is an even multiple of $\kappa(n - 1)$, we have $\beta(n) \mod \kappa(n - 1) = 0$.
  This implies $\beta(n) \in S_x(n - 1)$.
  Because $\beta$ monotonically increasing, change in $\beta$ strictly shrinks $S_x(n)$.

  \item $n \mod \kappa(n) = 0$

  Then the observation at time point $n$ is claimed for inclusion within $S_x(n)$.
  It is available, having just presently occurred.

\end{enumerate}

Note that if none of the above occur, then $S_x(n - 1) = S_x(n)$.
Any combination of the above maintains $S_x(n - 1) \in \{S_x(n), n\}$
% Theorem \ref{thm:geom-seq-nth-root-algo-self-consistency} details this point.


Why does this construction for target $n^{x/a}$ satisfy our $\mathcal{O}(1)$ space complexity constraint?
It can be shown that $|S_x(n)| \leq 6$.
This stems from cadence $\kappa$ as the binary floor of half $n^{x/a}$ (at most a quartering reduction) and backstop $\beta$ set at most $3/2 \times n^{x/a}$ time points back.

% Because cadence $\kappa$ is the binary floor of half $n^{x/a}$, we have $n^{x/a} \leq 4\kappa$.
% Recall that $\beta \geq n - 3 n^{x/a} / 2$.
% Remark that
% \begin{align*}
% |S_x(n)|
% = \lfloor \frac{n - \beta}{\kappa} \rfloor.
% \end{align*}

% Relaxing the integer floor,
% \begin{align*}
% |S_x(n)|
% \leq \frac{n - \beta}{\kappa}.
% \end{align*}

% Substituting inequalities,
% \begin{align*}
% |S_x(n)|
% &\leq 4\frac{n - \beta}{4\kappa}\\
% &\leq 4\frac{n - \beta}{n^{x/a}}
% \end{align*}

% and then

% \begin{align*}
% |S_x(n)|
% \leq 4\frac{n - \Big(n - 3 n^{x/a} / 2\Big)}{n^{x/a}} \\
% \leq 4\frac{3 n^{x/a} / 2}{n^{x/a}} \\
% \leq 4 \times \frac{3}{2} \\
% \leq 6.
% \end{align*}
% Theorem \ref{thm:geom-seq-nth-root-algo-space-complexity} details this point.

Why does this curated set for target $n - n^{x/a}$ satisfy our gap size bound $n^{x/a}$?
Because cadence $\kappa \leq \frac{n^{x/a}}{2}$, gap size satisfies the bound.
Because backstop $\beta \leq n - n ^{x/a}$, the target time point is covered within the cadenced range.
% Theorem \ref{thm:geom-seq-nth-root-algo-uncertainty-bound} details this point.

We bring curated sets for each target point together in a set union to produce the overall GSNR retained set $R$,
\begin{align*}
  R(n)
  &=
  \bigcup_{i=0}^{a} S_i(n).
\end{align*}
Note that under this construction, policy algorithm self-consistency and extant record size bounds follow from those shown for constructions for individual targets $n^{x/a}$.
\footnote{
A strict upper bound of $6a + 2$ for extant record size can be calculated, although we do not demonstrate it here.
}
Time points that should be dropped to enact the GSNR policy algorithm follow from set subtraction between $R(n)$ and $R(n+1)$.

We have discussed resolution guarantees for individual target point constructions, but what resolution guarantee is afforded overall for an arbitrary time point $t$ with age $g = n - t$?
For such a time point $t$, the tightest resolution guarantee is that of the next older target point.
Taking
\begin{align*}
\alpha = n^{1/a}
\end{align*}
the age of the next older target time point will be
\begin{align*}
\alpha^{ \lceil \log_{\alpha} g \rceil },
\end{align*}

By the target time point resolution guarantee we established earlier, the gap size provided at this target time point is bounded by its age.

Observe that, at most, the next older target time point age will be a factor of $\alpha = n^{1/a}$ greater than $g$.
So, the worst case absolute provided gap size is
\begin{align*}
\alpha \times g = n^{1/a} \times g.
\end{align*}

Worst case recency-proportional gap size is therefore $n^{1/a}$.

Figure \ref{fig:retention-policies} includes a time-lapse of the extant record under the GSNR policy algorithm.
The distinguishing feature of the GSNR policy algorithm is its sustained mitigation of worst recency-proportional gap size without unbounded record size growth.
% ELD: are we saying "effort" because its an in practice benefit but not a provable asymptotic one?
% MAM: trying to mean that it's asymptotic but not perfect
This lends it to very-long duration applications.
% However, before very-deep record depths, we have observed that recency-proportional resolution tends to better distribute equivalent numbers of retained observations.

% % TODO: missing proof
\begin{theorem}{Geometric Sequence $n$th Root Self-Consistency}
\label{thm:geom-seq-nth-root-algo-self-consistency}

\end{theorem}

\begin{proof}
\label{prf:geom-seq-nth-root-algo-self-consistency}

\end{proof}

% % TODO: missing proof
\begin{theorem}{Geometric Sequence $n$th Root Space Complexity}
\label{thm:geom-seq-nth-root-algo-space-complexity}
The \gls{extant record size} of the Geometric Sequence $n$th Root Policy Algorithm grows with order $\mathcal{\theta}{(k)}.$

\end{theorem}

\begin{proof}
\label{prf:geom-seq-nth-root-algo-space-complexity}



\end{proof}

% \begin{theorem}{Geometric Sequence $n$th Root Uncertainty Bound}
\label{thm:geom-seq-nth-root-algo-uncertainty-bound}

\end{theorem}

\begin{proof}
\label{prf:geom-seq-nth-root-algo-uncertainty-bound}

\end{proof}


% % TODO: missing pseudocode
\begin{algorithm}
\caption{Geometric Sequence $n$th Root Stratum Enumeration}
\label{alg:geom-seq-nth-root-gen-drop-ranks}
\begin{algorithmic}[1]
\State{Define $P:=T:=\{ \{1\},\ldots,\{d\}$\}}
\WHILE{$\#P > 1$}
\State{Find an optimal partition tree $T_{C^\prime}$ }
\State{Update $P := (P{\setminus} C^\prime) \cup \{ \bigcup_{t\in C^\prime} t \}$}
\State{Update $T := T \cup \{ \bigcup_{t\in\tau} t : \tau\in T_{C^\prime}{\setminus}
\mathcal{L}(T_{C^\prime})\}$}
\ENDWHILE
\Return $T$
\end{algorithmic}
\end{algorithm}

% % TODO: missing pseudocode
\begin{algorithm}
\caption{Geometric Sequence $n$th Root Stratum Retention Predicate}
\label{alg:geom-seq-nth-root-algo-pred-keep-rank}
% STUB ALGORITHM
\end{algorithm}

\begin{algorithm}
\caption{Geometric Sequence $n$th Root Stratum Enumeration}
\label{alg:geom-seq-nth-root-enum-retained-ranks}
\begin{algorithmic}
    \Require{ $\texttt{n}$ -- the number of strata deposited }
    \Require{ $\texttt{p}$ -- the interspersal  }
    \Require{ $\texttt{d}$ -- the degree }
    \Ensure{ set of retained strata }
\end{algorithmic}

\begin{algorithmic}[1]
    \State $\texttt{set} \gets {0, n - 1}$

    \For{$p = 1$ \textbf{to} $\texttt{d} + 1$}
        \State $\texttt{r} \gets n^{\frac{p}{d}}$

        \State $\texttt{k} \gets \text{largest integral power of two} \le \max \{\frac{r}{p}, 1 \}$ % retained_ranks_sep

        \State $\texttt{c} \gets \max \{n - \ceil{r + \frac{r}{p}}, 0\}$
        \State $\texttt{a} \gets \texttt{c} - (\texttt{c} \bmod -\texttt{k})$

        \State $\texttt{set} \gets \texttt{set} \cup \{\texttt{a}, \texttt{a} + \texttt{k}, \texttt{a} + 2\texttt{k}, {a} + 3\texttt{k}, \dots, n\}$
    \EndFor
\end{algorithmic}
\end{algorithm}



\subsection{Curbed Recency-proportional Resolution Algorithm}
\label{sec:curbed-recency-proportional-resolution-algo}

\begin{algorithm}
\caption{Curbed Recency-proportional Resolution Stratum Retention Predicate}
\label{alg:curbed-recency-proportional-algo-pred-keep-rank}
\begin{algorithmic}[1]
\STATE{Define $P:=T:=\{ \{1\},\ldots,\{d\}$\}}
\WHILE{$\#P > 1$}
\STATE\label{line3}{Choose $C^\prime\in\mathcal{C}_p(P)$ with $C^\prime :=
\operatorname{argmin}_{C\in\mathcal{C}_p(P)} \varrho(C)$}
\STATE{Find an optimal partition tree $T_{C^\prime}$ }
\STATE{Update $P := (P{\setminus} C^\prime) \cup \{ \bigcup_{t\in C^\prime} t \}$}
\STATE{Update $T := T \cup \{ \bigcup_{t\in\tau} t : \tau\in T_{C^\prime}{\setminus}
\mathcal{L}(T_{C^\prime})\}$}
\ENDWHILE
\RETURN $T$
\end{algorithmic}
\end{algorithm}


