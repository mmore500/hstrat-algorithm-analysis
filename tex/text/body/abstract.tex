\begin{abstract}
Data stream algorithms tackle operations on high-volume sequences of read-once data items.
Data stream scenarios include inherently real-time systems like sensor networks and financial markets.
They also arise in purely-computational scenarios like ordered traversal of big data or long-running iterative simulations.
In this work, we develop methods to maintain running archives of stream data that are temporally representative, a task we call ``stream curation.''
Our approach contributes to rich existing literature on data stream binning, which we extend by providing stateless (i.e., non-iterative) curation schemes that enable key optimizations to trim archive storage overhead and streamline processing of incoming observations.
We also broaden support to cover new trade-offs between curated archive size and temporal coverage.
We present a suite of five stream curation algorithms that span $\mathcal{O}(n)$, $\mathcal{O}(\log n)$, and $\mathcal{O}(1)$ orders of growth for retained data items.
Within each order of growth, algorithms are provided to maintain even coverage across history or bias coverage toward more recent time points.
Our exposition treats stream curation in the context of application to a recently-introduced technique for distributed provenance tracking, which we call ``hereditary stratigraphy.''
Hereditary stratigraphy enables post hoc reconstruction of ancestry trees for digital artifacts without any centralized tracking.
Presented stream curation algorithms serve at the core of hereditary stratigraphy to provide explicit, highly tunable trade-offs between memory use and reconstruction accuracy.
We propose a new reconstruction technique for hereditary stratigraphy, which demonstrates how curated stream data can efficiently reconstruct ancestry trees for large artifact populations.
Such ancestry trees can benefit not only parallel and distributed evolutionary computation/simulation, but also other scenarios where digital artifacts replicate without centralized observability such as computer viruses or content in distributed social media networks.
More broadly, memory-efficient stream curation can boost the data stream mining capabilities of low-grade hardware in roles such as sensor nodes and data logging devices.
\end{abstract}
