\begin{abstract}

% TODO we should probably use the word near optimal somewhere

Reconstruction of phylogenetic history from natural biosequences kindles an evergreen profusion of statistical and algorithmic innovation because it is interesting and useful but especially because it is hard.
Efforts to observe the phylogenetic dynamics of highly distributed agent-based evolutionary models have motivated a peculiar converse question: how to best engineer artificial digital ``biosequences’’ to facilitate fast and accurate lineage reconstructions?
This issue directly implicates the design of virtual annotations on digital genomes to facilitate post-hoc inference — and analysis — of their lineages.
Emerging so-called ``hereditary stratigraphy’’ methodology abstracts this problem to a more general one: how to satisfactorily maintain a temporally representative collection of observations that stream in on a rolling basis.
We term this more general question the ``stream curation problem.’’
Because ongoing data feeds necessitate continual downsampling when faced with storage and memory limitations, the streaming curation problem occurs also in context of unattended data loggers, irregularly-uplinked wireless sensor network devices, and servers under continuous application monitoring, among other scenarios.
To meet requirements of diverse streaming curation scenarios, we present a suite of streaming curation policy algorithms that trade off along a continuum of collection size and temporal coverage.
Policies span $\mathcal{O}(n)$, $\mathcal{O}(\log n)$, and $\mathcal{O}(1)$ orders of growth in curated collection size.
Within each order of growth, policy algorithms differ in relative prioritization of newer versus older data.
We explore two alternatives: even density distribution across history or recency-proportional distribution with greater density at more recent time points.
Policy algorithm designs explicitly pre-determine retained collection composition at each time point, enabling key optimizations to reduce storage overhead and streamline processing of incoming observations.

Hereditary stratigraphy’s application of streaming curation to lineage inference arises through interpretation of each individual as an independent observation of lineage association within an ongoing generational stream.
Replicator annotations operationalize this abstract interpretation.
Each replicator is adjoined with a heritable set of checkpoint identifiers.
New checkpoints created each generation, and streaming curation algorithms decide which to retain.
Recent experimental work applying this principle has yielded robust recovery of phylogenetic information.
Here, we formalize the theoretical foundations of hereditary stratigraphy through exposition of streaming curation algorithms.
% and resolve a lynchpin limitation to its end-to-end application: methods for scalable, whole-tree reconstruction from checkpoint annotations.
As a point of comparison, we also provide formal analysis for existing centralized methods for direct phylogenetic tracking.
Phylogenetic analysis capabilities significantly advance distributed agent-based simulations as a tool for evolutionary research.
Such decentralized tracing could extend also to other digital artifacts that proliferate through replication, like digital media and computer viruses.
Implications of underlying streaming curation algorithms extend more broadly still, encompassing wider issues of rolling record management.
\end{abstract}
