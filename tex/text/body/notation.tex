\section{Preliminaries, Notations, and Terminology} \label{sec:notation}

The core function of proposed algorithms ``stream curation,'' to dynamically filter out a subset of incoming data that ensures the structure of gaps in history created by discarding items respects some desiderata.
Be reminded that this framing considers data items solely with respect to their logical sequence position, and disregard their actual semantic values.
Incoming data is assumed to arrive on a rolling basis, as a \textbf{data stream} comprised of sequential \textbf{data items} $v_i$.
We presume the data stream to be transient (i.e., "read once"), and describe the process of reading an item from the data stream as ingesting it.

This section lays out key notations used throughout this work, summarized in Table \ref{tab:notation}.

\begin{table}[]
\begin{tabular}{lllll}
\hline
\textit{Description} & \textit{Type} & \textit{Notation} & \textit{Definition} & \textit{Domain} \\ \hline
\rowcolor{gray!20}
\multicolumn{5}{c}{\textbf{Time}} \\ \hline
Current Time & int & $\colorT$ & num elapsed data item ingests & $\in \mathbb{N}$ \\
Data Item Ingest Time & int & $\colorTbar$ & num ingests preceding data item & $\in [0 \twodots \colorT)$ \\ \hline
\rowcolor{gray!20}
\multicolumn{5}{c}{\textbf{Curation Policy}} \\ \hline
Kept Item Enumeration & seq & $\colorK(\colorT)$ & algorithm-defined & $\subseteq [0 \twodots \colorT)$ \\
\makecell[l]{Dropped Item Enumeration\\~} & \makecell[l]{seq\\~} & \makecell[l]{$\colorD(\colorT)$\\~} &  \makecell[l]{items discarded while ingesting $\colorTbar=\colorT$\\ i.e., \footnotesize $\colorK(\colorT + 1) \setminus ( \{\colorT + 1\} \cup \colorK(\colorT + 1) )$} & \makecell[l]{$\subseteq [0 \twodots \colorT)$\\~} \\ \hline
\rowcolor{gray!20}
\multicolumn{5}{c}{\textbf{Curation Quality}} \\ \hline
Discarded Data Items & set & $\colorBnot_{\colorT}$ & $[0\twodots\colorT) \setminus \colorK(\colorT)$ & $\subseteq [0\twodots\colorT)$ \\
Gap Size in Curated Collection & int & $\colorg = \colorG_{\colorT}(\colorTbar)$ & Formula \ref{eqn:gap-size-defn} & $\in [0 \twodots \colorT)$
\end{tabular}
\centering
\caption{
Summary of notation used.
}
\label{tab:notation}
\end{table}

\subsection{Logical Time $\colorT$ and Item Ingest Time $\colorTbar$}
\label{sec:notation-time}

We will refer to each data item's stream sequence index as its \textbf{ingest time} $\colorTbar$ and the number of items ingested as the \textbf{current logical time} $\colorT$.
In other contexts, a data item's ingest time $\colorTbar$ might be referred to as its ``sequence position'' within the data stream.
However, we avoid that terminology to prevent confusion of sequence position with buffer position $\colork$.

We use a zero-indexing convention.
Logical time begins at $\colorT=0$, when no data items have yet been ingested.
The first element of the data stream $v_0$ is assigned ingestion time $\colorTbar=0$.
After the first item $v_0$ is ingested, logical time advances to $\colorT=1$.
We assume $\colorT$ to be known at every point, which can be accomplished trivially in practice with a simple counter.
Because we are only concerned with the sequence order of data items (and not their actual data values), we will shorthand $\colorTbar$ as referring to $v_{\colorTbar}$ (i.e., the data item ingested at that time).

\subsection{Dropped Item Enumeration $\colorD(\colorT)$}
\label{sec:dropped-item-enumeration}

Each time a new data item $\colorTbar = \colorT$ is ingested, the active policy algorithm decides which, if any, of retained data items $\colorTbar < \colorT$ should be discarded.
We refer to this as the set of \textbf{dropped items} $\colorD(\colorT)$.
By definition, $\colorD(\colorT) \subseteq [0\twodots\colorT)$,

Note that we do not allow items to be dropped multiple times.
That is, for any $\colorT_1, \; \colorT_2 \in \mathbb{N}$ where $\colorT_1 \neq \colorT_2$, we require that $\colorD(\colorT_1) \cap \colorD(\colorT_2) = \emptyset$.
We call this property \textbf{self-consistency}, meaning that items can not be dropped more than once -- once they're gone, they're gone.
We will prove self-consistency on an algorithm-by-algorithm basis, although our  focus in analyzing self-consistency will be in terms of kept items, introduced next.

\subsection{Kept Item Enumeration $\colorK(\colorT)$}
\label{sec:kept-item-enumeration}

We define \textbf{kept items} $\colorK(\colorT)$ as the set of items ingested but not discarded.
Formally,
\begin{align*}
\colorK(\colorT)
= [0\twodots\colorT) \setminus \bigcup_{\colorT' \in [0\twodots \colorT)} \colorD(\colorT').
\end{align*}

\textbf{Self-consistency} among kept items requires that if an item $\colorTbar$ is to be kept at time $\colorT$, it must have been kept at all time points $\colorT' \geq \colorT$.
This is because once a data item is discarded, it cannot be recovered.
Formally, for any $0 \leq \colorT_1 < \colorT_2$,
\begin{align*}
\colorK(\colorT_2)
\subseteq
\colorK(\colorT_1) \cup [\colorT_1 \twodots \colorT_2).
\end{align*}

\subsection{Application in Practice}
%TODO move elsewhere?

While we provide algorithms implementing both $\colorK(\colorT)$ and $\colorD(\colorT)$ for all algoriths, we analyze algorithms primarily in terms of $\colorK(\colorT)$.
However, the availability of both algorithms is important.
Having $\colorK$ and $\colorD$ allows data items to be managed stored contiguously in an array in chronological order without timestamps.
This can be important for data items that are very small, where the overhead of timestamps or other data structure (e.g., pointers) would be substantial.
In this case, $\colorK(\colorT)$ is essential for identifying stored data items.
Alternately, using just $\colorD(\colorT)$ the data items can be stored in a hash map.
Under such a scheme, data items would be stored with the mapping $\colorTbar \to v_{\colorTbar}$.
Then, each time step, the newly ingested item $\colorTbar = \colorT$ would be stored and dropped data items $\colorD(\colorT)$ would be deleted from the hash map.

\subsection{Gap Size $\colorg$}
\label{sec:notation-gapsize}

% Be reminded that \textit{coverage criteria} for retained data items considered here operate solely in terms of items' time indices $\colorTbar$.
We define coverage criteria in terms of \textbf{gap sizes} in the retained record.
Formally, we define gap size as a count of consecutive data items that have been discarded or overwritten.
Let $\colorBnot_{\colorT}$ refer to data items discarded up to that point $\colorT$.
Gap size for record index $\colorTbar \in [0 \twodots \colorT)$ at time $\colorT$ follows as
\begin{align}
\colorG_{\colorT}(\colorTbar)
&\coloneq
\max
\{
  i + j
  \text{ for }
  i,\;\; j \in \mathbb{N}
  :
  [\colorTbar-i \twodots \colorTbar+j) \subseteq \colorBnot_{\colorT}
\}.
\label{eqn:gap-size-defn}
\end{align}
Note that if $\colorTbar \in \colorK(\colorT)$, then $\colorG_{\colorT}(\colorTbar) = 0$.

\subsection{Miscellania}

Algorithm listings refer to a handful of utility helper functions (e.g., \textsc{BitFloor}, \textsc{BitCeil}, etc.).
Refer to Supplementary Section \ref{sec:pseudocode} for full definitions of these.

We denote the binary floor of a value $x$ as $\left\lfloor x \right\rfloor_\mathrm{bin} = 2^{\left\lfloor \log_2 x \right\rfloor}$.
For binary ceiling, $\left\lceil x \right\rceil_\mathrm{bin} = 2^{\left\lceil \log_2 x \right\rceil}$.
Take $\left\lfloor x \right\rfloor_\mathrm{bin} = \left\lceil x \right\rceil_\mathrm{bin} = 0$.
Finally, take $\{2^{\mathbb{N}}\}$ as shorthand for $\{2^n : n \in \mathbb{N} \}$.
