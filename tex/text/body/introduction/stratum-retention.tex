\subsection{Stratum Retention}

Putting ``hereditary stratigraphy'' techniques into practice requires a point of further consideration.
Naive application of hereditary stratigraphy would produce indefinitely lengthening genome annotation size in direct proportion to generations elapsed.
To overcome this obstacle, hereditary stratigraphy prescribes a ``pruning'' process to discard strata on the fly as new strata are deposited.
Pruning reduces annotation size, but at the expense of introducing inferential uncertainty: the last generation of common ancestry between two lineages can be resolved no finer than the time points associated with retained strata.
Deciding which strata to discard is the main point of algorithmic interest associated with the technique.
In fact, this scenario turns out to be an instance of a more general class of online algorithmic problems we hereby term as ``streaming curation.''
In the context of hereditary stratigraphy, we refer to the decision-making rules that decide which retained strata to discard per generation as a ``stratum retention'' algorithm.

The obvious core issue of stratum retention is how many strata to discard.
In many applied cases, it is desirable to keep the count of retained strata at or below a size cap indefinitely as generations elapse.
However, in some cases discussed later on, it may be desirable to allow a logarithmic growth rate of annotation size to guarantee upper bounds on inferential uncertainty.

Determining stratum retention strategy also raises a more subtle second consideration: what skew, if any, to induce on the composition of retained strata.
Strata from evenly-spaced time points may be retained in order to provide uniform inferential detail over the entire range of elapsed time points.
However, coalescent theory predicts that evolution-like processes will tend to produce phylogenies with many recent branches and progressively fewer more ancient branches \citep{nordborgCoalescentTheory2019, berestyckiRecentProgressCoalescent2009}.
Thus, fine inferential detail over more recent time points is usually more informative to phylogenetic reconstruction than fine detail over more ancient time points.
Thus, among a fixed-size retained sampling of time points, skewing the composition of retained strata to over-represent more recent time points would likely provide better bang for the buck with respect to reconstructive power.
Indeed, experiments reconstructing known lineages have shown that recency-skewing retention provides better quality reconstructions \citep{moreno2022hereditary}.
So, in addition to evenly-spaced retention, we consider retention allocations that yield gap widths between successive retained strata (and corresponding estimation uncertainty) scaled proportionately to those strata's depth back in history.

Because no single retention policy can meet requirements of all use cases, we present in Section \ref{sec:annotation-algorithms} a suite of complementary stratum retention algorithms covering possible combinations of retained stratum count order of growth and chronological skew.
We describe retention algorithms in terms of upper bounds on memory usage and inference uncertainty.

With respect to memory usage, we refer to guaranteed upper bounds as ``size order of growth'' in the asymptotic case with respect to elapsed generations or ``size cap'' in the absolute case.
We refer to bounds on spacing between retained strata a ``resolution guarantee.''
Resolution guarantee specification incorporates both the total number of generations elapsed and the historical depth of a particular time point in the stratigraphic record --- so, bounding is tailored within these particular circumstances.

%ELD: TODO: This said "three nutes and bolts consdierations" but there were only two. I changed it to say two, but is there one missing?
In addition to their particular size bounds and resolution guarantees, we demonstrate all proposed stratum retention algorithms satisfy two nuts and bolts algorithmic considerations:
\begin{enumerate}
\item \textbf{Tractability of directly enumerating deposition time of retained strata at any arbitrary generation.} Efficient computation of the deposition times retained at each time point provides a tractable reverse mapping from column array index to deposition generation.
  Such a mapping enables deposition generation to be omitted from stored strata, potentially yielding several-fold space savings (depending on the differentia bit width used).

\item \textbf{Capability to enumerate discarded strata in constant time}

\item \textbf{Stratum discard sequencing for ``self-consistency.''}
  When you discard a stratum now, it won't be available later.
  If you need a stratum at a particular time point, you must be able to guarantee it hasn't already been discarded at any prior time point.

\end{enumerate}

Note that size bounds and resolution guarantees are enforced across all generations.
Thus, stratum retention policies must manage ``online'' column composition across rolling generations.
Indeed, for many use cases, resolution and column size guarantees will need to hold at all generations because the number of generations elapsed at the end of an experiment is often not known \textit{a priori} and the option of continuing a fixed-length experiment with evolved genomes is desired.
This factor introduces a design subtlety: as generations elapse, deposited strata recede to increasingly ancient historical depth with respect to the current generation.
Resolution guarantees may change along the way back.
In those cases, cohorts of retained strata must, in dwindling, gracefully morph through a constrained series of retention patterns.
