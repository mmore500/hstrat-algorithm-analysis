\subsection{Streaming Curation}

Memory-efficient representation developed for hereditary stratigraphy can benefit other applications of stream curation and more general binning on data streams.
Indeed, hardware trends promise to expand operational scenarios requiring memory-critical data stream processing.
High-performance computing (HPC) expects to continue emphasis of scale-out with lean processing cores rather than boosting capacities of individual processing components \citep{sutter2005free,morgenstern2021unparalleled}.
The Cerebras Wafer-Scale Engine epitomizes this trend, packaging an astounding 850,000 computing elements onto a single die.
Individual core, however, host just 48kb of memory and operate through a exclusively-local mesh communication model \citep{cerebras2021wafer,lauterbach2021path}.

As with HPC, component economization and minaturization influenced the Internet of Things (IoT) revolution \citep{rfc7228,ojo2018review}, a march potentially culminating in ``smart dust'' \citep{warneke2001smart} of downscale, low-end hardware.
The Michigan Micro Mote platform for instance, provisions a mere 3kb of retentive memory within its cubic millimeter form factor \citep{lee2012modular}.
More recent work has explored devices tucked within the form factor of dandelion parachutes \citep{iyer2022wind}.
The chipset is yet more austere, provisioning 2 kilobytes of volatile flash memory --- and a mere 128 bytes of retentive memory \citep{microchip2014atiny20}.
As engineers continue to plumb the extremities of technical feasibility, bare-bones computing modalities will persist, and applications in wireless sensor networks will necessitate lightweight data stream algorithms.

Stepping back, the online filtering obligations faced by hereditary stratigraph annotations are not unlike those faced by unattended data logger devices.
Both manage incoming observation streams on a potentially indefinite or indeterminate basis and both operate under storage space limitations.
Further, both are presumedly tasked to operate under some stipulation for time coverage, whether simply rolling full retention of most recent data within available buffer space \citep{fincham1995use}, dismissal of incoming data after storage reaches capacity \citep{saunders1989portable,mahzan2017design}, best-effort even coverage of the elapsed period, or otherwise.
Even high-capacity devices may experience overflow conditions when confronted with high-frequency data streams \citep{luharuka2003design}.

Restated explicitly, these scenarios confront a question of how to satisfactorily maintain a temporally representative cross section of observations that stream in on a rolling basis.
We propose this problem as the ``stream curation problem.''
For the sake of generalizability, Section \ref{sec:annotation-algorithms} organizes presentation of stratum retention algorithms through the lens of the streaming curation problem.
There has been some work to extend the record capacity of data loggers through application-specific online compression algorithms \citep{hadiatna2016design}, but to our knowledge the streaming curation problem has not been treated directly from either a theoretical or empirical vantage.


Some of the same curation policy algorithms we propose for stratum retention could also be useful in these cases.
For example, organization of IoT devices into wireless sensor networks is expected, in a considerable fraction of cases, to structure irregular device uplink schedules, such as the ``mobile sink'' paradigm \citep{jain2022survey}.
Under this model, network base station(s) physically traverse the coverage area and transact with nearby sensor nodes.
Reliance on the mobile sink's patrol schedule potentially introduces uncertainty in data transfer schedules.

Recency-proportional retention may suit some applications, where time intervals of interest may be flagged well after the fact, but tend to bias toward the recent past.
Finally, streaming curation may even pertain to record management in large capacity centralized storage systems in some scenarios \citep{bhat2018data}.


%% TODO continue revisions

Lightweight

There has been some work to extend the record capacity of data loggers through application-specific online compression algorithms \citep{hadiatna2016design}, but to our knowledge the streaming curation problem has not been treated directly from either a theoretical or empirical vantage.




A common question being whether to distribute bin density evenly over history or to bias information to recent events.

running or rolling, where you just consider the latest data, and binning, where you group and summarization
Among data streOne common device.
Across data stream algorithms, common operations include A, B, and C


windowing (tilted-time) natural time \citep{giannella2003mining}

dimension-reduction \citep{zhao2005generalized}
amnesic functions \citep{palpanas2008streaming}
We refer to it as the stream curation problem,


@article{zhao2005generalized,
  title     = {Generalized dimension-reduction framework for recent-biased time series analysis},
  author    = {Zhao, Yanchang and Zhang, Shichao},
  journal   = {IEEE Transactions on Knowledge and Data Engineering},
  volume    = {18},
  number    = {2},
  pages     = {231--244},
  year      = {2005},
  publisher = {IEEE}
}
@inproceedings{palpanas2004online,
  title        = {Online amnesic approximation of streaming time series},
  author       = {Palpanas, Themistoklis and Vlachos, Michail and Keogh, Eamonn and Gunopulos, Dimitrios and Truppel, Wagner},
  booktitle    = {Proceedings. 20th International Conference on Data Engineering},
  pages        = {339--349},
  year         = {2004},
  organization = {IEEE}
}
@article{palpanas2008streaming,
  title     = {Streaming time series summarization using user-defined amnesic functions},
  author    = {Palpanas, Themis and Vlachos, Michail and Keogh, Eamonn and Gunopulos, Dimitrios},
  journal   = {IEEE Transactions on Knowledge and Data Engineering},
  volume    = {20},
  number    = {7},
  pages     = {992--1006},
  year      = {2008},
  publisher = {IEEE}
}
@inproceedings{aggarwal2003framework,
  title        = {A framework for clustering evolving data streams},
  author       = {Aggarwal, Charu C and Philip, S Yu and Han, Jiawei and Wang, Jianyong},
  booktitle    = {Proceedings 2003 VLDB conference},
  pages        = {81--92},
  year         = {2003},
  organization = {Elsevier}
}

We provide C++ and Python implementations in the accompanying \texttt{hstrat} library \citep{moreno2022hstrat}.


Early days of computing were marked by resource scarcity.
As desktop computing has advanced, this becomes less salient.

For the sake of generalizability, Section \ref{sec:annotation-algorithms} organizes presentation of stratum retention algorithms through the lens of the streaming curation problem.

% autocannabilism https://epubs.siam.org/doi/pdf/10.1137/1.9781611972795.8
% visual sedimentation https://ieeexplore.ieee.org/stamp/stamp.jsp?tp=&arnumber=6634152

Closely related to tilted-time sampling \citep{giannella2003mining,han2005stream}

\citep{zhao2005generalized} <-- merging windows

\citep{palpanas2004online} <-- custom amnesiac function
