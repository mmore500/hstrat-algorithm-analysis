\section{Perfect Tracking Algorithm} \label{sec:perfect-tracking}

Traditionally, the problem of recording the phylogenies of replicating digital 
artifacts has been solved using a ``perfect tracking'' approach where all replication 
events are recorded and stored in a tree (or forest, if there are multiple roots) \citep{dolson2023phylotrackpy}.
In this section, we present results on the time and memory size complexity of perfect tracking;
these results will serve as a basis of comparison for our curation policy algorithms in the context
of lineage tracking.

Perfect tracking algorithms are designed to plug into the ongoing replication proces.
As such, they exist in three separate steps: 1) initialization, 2) handling the birth of new objects, and 3) (optionally) handling the removal of existing objects.
The initialization step happens at the beginning of the process being measured.
The birth handling sub-algorithm gets called any time a new object is created.
Similarly, the removal handling sub-algorithm gets called any time an object is removed from the set of currently active objects.

Perfect tracking can be used regardless of whether each artifact has one or multiple parents (i.e. asexual vs. sexual reproduction).
Our analysis here will focus on the case where each artifact has only parent, but most of our results generalize to the multi-parent case.


\subsection{Naive perfect tracking}

The naive perfect tracking algorithm is fairly straightforward and is formally desribed in algorithm \ref{alg:perfect-tracking}. 

\begin{algorithm}
    \caption{Perfect phylogenetic tracking}
    \label{alg:perfect-tracking}
    \begin{algorithmic}[1]
    \STATE{Define $T:=$ an empty tree}
    \STATE{Define $P:=$ a population of size $n$}
    \STATE{Add node representing common ancestor to $T$}
    \FOR{$i = 0$ to $g$}
        \FOR{$j = 0$ to $n$}
            \STATE{Add node representing $P_j$ to $T$}
            \STATE{Add edge in $T$ from parent of $P_j$ to $P_j$}
        \ENDFOR
    \ENDFOR
    \RETURN $T$
    \end{algorithmic}
\end{algorithm}

\subsubsection{Time complexity}

The naive perfect tracking algorithm can be implemented in constant time (see theorm \ref{thm:perfect-tracking-time}).

\begin{theorem}{Naive Perfect Tracking Time Complexity}
\label{thm:perfect-tracking-time}
The naive perfect tracking algorithm can be implemented in constant time ($\mathcal{O}(1)$) per birth event.
\end{theorem}

\begin{proof}
\label{prf:perfect-tracking-time}
For the purposes of this proof, we use the RAM model of computation. 
Initialization takes a single simple operation and only happens once. 
Thus, it is trivially constant-time.
Handling a birth event takes two simple operations.
Thus, each run of the birth-handling algorithim trivially takes constant time.
The birth-handling algorithm will run once per birth event.
Although the number of birth events is likely not constant with respect to the size of the population or the number of generations,
these birth events are not part of the tracking algorithm and would have happened regardless of whether phylogeny tracking was in place.
Consequently, adding phylogeny tracking will only increase the time complexity of a computational process to which it is added by a constant amount per birth event.
\end{proof}



\subsubsection{Size order of growth}

Naive perfect tracking has a very large size order of growth, owing to the fact that it maintains a tree containing a node for every object that ever existed (see theorem \ref{thm:perfect-tracking-space}).

\section{Additional notes on the memory requirements of perfect tracking} \label{sec:perfect-tracking-space-supp}
% TODO: section name that isn't monstorously long

Note that perfect phylogeny tracking algorithms are designed to plug into a larger computational process.
For phylogeny tracking to be relevant, this computational process must have a collection of $N$ objects that are currently eligilble to be copied (in evolutionary terms, the population).
Therefore, the space complexity of this process must be at least $\mathcal{O}(N)$.
As long as the expected size of $T$ scales no faster than $\mathcal{O}(N)$, then, phylogeny tracking will not be the primary factor determining memory usage.
While the memory cost of phylogeny tracking could still be significant in these cases, it is unlikely to be the primary factor determining whether running the program is tractable.

Previously, we noted that cases where perfect tracking with pruning achieves its worst case size order of growth are esoteric.
There are two ways that theses situations can occur:

1. $N$ is continuously increasing.
2. There is effectively no coalescence.



\subsection{Perfect tracking with pruning}

For most substantial applications, the amount of memory used by naive perfect tracking is prohibitive.
Instead, most practitioners use an alternative perfect tracking algorithm that includes a ``pruning'' step to reduce memory use (see algorithm \ref{alg:perfect-tracking-pruning}).
This algorithm removes all parts of the phylogeny that are no longer relevant to explaining the history of currently extant objects.
Thus, like naive perfect tracking, it introduces no uncertainty into down-stream analyses.

\begin{algorithm}
    \caption{Perfect phylogenetic tracking with pruning}
    \label{alg:perfect-tracking-pruning}
    \textbf{Part 1: Initialization} \\
    Same as algorithm \ref{alg:perfect-tracking} \\
    \textbf{Part 2: Handle birth event}
    \begin{algorithmic}[1]
        \REQUIRE{$P_i$ = newly created organism, $P_j$ = $P_i$'s parent}
        \STATE{Add node representing $P_i$ to $T$}
        \STATE{Add edge in $T$ from $P_i$ to $P_j$}
        \STATE{Record that $P_i$ is alive}
        \STATE{Increment $P_j$'s count of living offspring lineages}
    \end{algorithmic}
    \textbf{Part 3: Handle removal event}
    \begin{algorithmic}[1]
        \REQUIRE{$P_i$ = newly removed organism}
        \STATE{$P_j$ $\gets$ parent of $P_i$}
        \STATE{Record that $P_i$ is not alive}

        \WHILE{$P_i$ is not alive \textbf{and} $P_i$'s count of living offspring lineages == 0}
            \STATE{Remove $P_i$ from $T$}
            \STATE{Decrement $P_j$'s count of living offspring lineages}
            \STATE{$P_i$ $\gets$ $P_j$}
            \STATE{$P_j$ $\gets$ parent of $P_j$}
        \ENDWHILE

    \end{algorithmic}
\end{algorithm}

\subsubsection{Time Complexity}

The pruning algorithm has a constant (amortized) time complexity (see theorem \ref{thm:perfect-tracking-pruning-time}). 

\begin{theorem}{Pruning Time Complexity}
\label{thm:perfect-tracking-with-pruning-time}
The time complexity of pruning in perfect phylogenetic tracking is $\mathcal{O}(1)$, amortized. 
\end{theorem}

\begin{proof}
\label{prf:perfect-tracking-with-pruning-time}
The initialization and birth-handling steps of the pruning algorithm are trivially constant-time, as explained in proof \ref{prf:perfect-tracking-time} (technically the pruning algorithm adds two constant-time operations to the birth-handling step, but this fact is inconsequential).
Next, we prove that the removal-handling pruning algorithm runs in amortized constant time.
When an arbitrary taxon $t$ is removed, there are three possibilities: 1) $t$ has living descendants, 2) $t$ has no living descendants, but $t$'s parent is either alive or has other descendants that are alive, or 3) $t$ was the last living descendant of its parent.
In case 1, nothing can be removed from the phylogeny.
Case 1 takes contant time, because it only requires executing a single comparison operation.
In case 2, only $t$ can be removed from the phylogeny.
Case 2 also takes constant time, as it only requires removing a single node from the tree and doing three comparison operations.

In case 3, pruning must be done.
We recurse back up the lineage until we find a taxon, $a$, that is either alive or has living descendants. 
Let the distance between $t$ and the root of the tree be $d_t$ and the distance between $a$ and the root of the tree be $d_a$.
The pruning operation, then, takes $\mathcal{O}(d_t - d_a)$ steps. 
In the worst case, this value will be equal to the number of elapsed generations.

However, for case 3 to occur, all taxa from $t$ to $a$ (including $a$ but not including $t$) must have already been removed.
Consequently, $d_t - d_a$ case 1 removal operations must happen for every case 3 removal operation.
Thus, the amortized time to remove the sequence of taxa from $t$ to $a$ (including both $t$ and $a$) is:

\[
\frac{2(d_t - d_a) + 1}{d_t - d_a + 1}
\]

Since

\[
\lim_{d_t - d_a\to\infty} \frac{2(d_t - d_a) + 1}{d_t - d_a + 1} = 2
\],

the amortized time complexity of pruning is bounded by a constant, i.e. it is $\mathcal{O}(1)$.


\end{proof}

% TODO: missing proof
% \begin{theorem}{Average Space Complexity of Perfect Tracking with Pruning}
% \label{thm:perfect-tracking-with-pruning-space}
% Average case is best described by the upper-case-lambda coalescent, but math on that is maybe
% less well understood than Kingman coalescent. Kingman coalescent makes a number of more limiting
% assumptions, but most of them are actually fairly appropriate for most phylogeny tracking situations
% (fixed-size population, wright-fisher model). The biggest problem is that it assumes only bifurcations
% (no multi-furcations). However, a strictly bifurcating tree is the worst case for average space complexity
% (a series of non-bifurcating trees is the absolute worst case, but under random selection it would
% quickly turn into a pretty good (best?) case). So, if we can't do an exact proof based on the upper
% case lambda coalescent, we can still do worst average case proof based on the Kingman coalescent
% Oh wait, maybe bifurcations aren't the worst case? Apparently there is a paradoxical result that shows
% that the Kingman Coalescent decreases fastest???

% Okay, here's the thing - all of these different models behave in ways that populations being tracked
% might behave. The argument to make here is that for a ton of them, the size of the tree (Ln) has been
% shown to be constant, proportional to n, limited by some distribution or asymptote, etc. It may even
% be sufficient to say that they come down from infinity. We end up with a space complexity something like
% O(generations + Ln) which in the worst case could translate to something like O(generations + 2*pop_size^2),
% except I think we can actually prove it's actually something more like O(generations + pop_size + log(pop_size))
% slight risk that pop_size should be multiplied by log(pop_size)

% \end{theorem}

% \begin{proof}
% \label{prf:perfect-tracking-with-pruning-space}

% \end{proof}

\subsubsection{Size order of growth}

As with naive perfect tracking, the primary memory cost of this algorithm comes from maintaining the tree of records, $T$.
Thus, our analysis in this section will focus on the size of $T$.
Technically, the worst-case size order of growth of perfect tracking with pruning is the same as for naive perfect tracking (see theorem \ref{thm:perfect-tracking-space}).
However, whereas that order of growth is also the best case for naive perfect tracking, it is a somewhat pathological case with pruning.
When using pruning, such a case would only occur when every object is copied before being removed.
There are some legitimate reasons these cases might occur, but most of them are fairly esoteric (see \ref{sec:perfect-tracking-space-supp}).

% These scenarios might legitimately occur if objects are never removed or the number of objects is allowed to continuously grow, but in these cases $N$ would also be growing continuously.
% They could potentially also happen when tracking digital objects that are copied with very small amounts of stochasticity.

When pruning is used, the asymptotic behavior of the average case is both substantially different from and more practically informative than the asymptotic behavior of the worst case.
Here, we investigate the expected size of $T$ (which we will call $E(|T|)$) in the average case as the population size ($N$) and number of time points ($G$) increase.
Of course, investigating the average case requires making some assumptions about the scenarios where this algorithm may be used.
An expedient assumption to make is that objects to be replicated are selected at random from the population.
While this assumption is obviously not completely true in most cases, it will turn out that $E(|T|)$ is actually smaller for most realistic cases.

Conveniently, the asymptotic behavior of $E(|T|)$ as $N$ increases has been studied under a variety of models of random selection.
The relevant branch of mathematics is coalescence theory, which describes the way particles governed by random processes come together over time [cite Berestycki].
While different models of random selection yield different behavior, nearly models that are realistic for our purposes exhibit ``coalesence'' events in which the most recent common ancestor of the entire current population becomes more recent as a result of old lineages dying out.
This observation implies that when we run our perfect tracking with pruning algorithm for sufficiently long periods of time, on average we should expect our tree to have a long ``stem'' of nodes with a single parent and single child.
The most recent node in this chain will be the most recent common ancestor (MRCA) of the whole extant population.
The descendants of the MRCA will form a more traditional looking tree, which we will call $T_c$.

The expected size of $T_c$ (denoted $E(|T_c|)$; often referred to as $L_n$ in coalescence literature), then, will be the dominant factor affecting $E(|T|)$.
Although the asymptotic behavior of $E(|T_c|)$ depends on the specifics of the evolutionary process, it scales sub-linearly with respect to population size for all realistic models that have been investigated thus far.
How is this possible?
Under realistic models, various subsets of the population will coalesce to common ancestors much faster than the whole population does.
Consequently, there are many shallow branches. 
The farther back in $T_c$ you go, the fewer branches there are.
Note that the time to coalescence depends entirely on the population size ($N$); the number of generations ($G$) does not affect the asymptotic growth rate of $E(|T_c|)$.

The precise asymptotic scaling of $E(|T_c|)$ depends on the exact model of random replication used.
Under one common model, Kingman's Coalescent, it is $\mathcal{O}(log(N))$.
Under another reasonable model, the Bolthausen-Sznitman coalescent, it is $\mathcal{O}(\frac{N}{log(N)})$.
%Notably, these are both less than $\mathcal{O}(N)$, which would necessarily be the space complexity of whatever program was generating the phylogeny being tracked.
%Thus, we can conclude that in the average case perfect tracking with pruning will not be the primary factor determining memory usage.
Some questionably-realistic models could also produce the star-shaped coalescent, in which $N$ offspring descend directly from the most recent common ancestor.
While this scenario actually takes less memory in total, its asymptotic growth is $\mathcal{O}(N)$.
Thus, we can claim a loose bound on the average case growth rate of $\mathcal{O}(N + G)$.
Trivially, the ``stem'' part of $T$ can be pruned off when coalescence occurs, producing a bound of $\mathcal{O}(N)$.

Earlier, we mentioned that in practice $T$ will usually be smaller than predicted by these random models.
These random models represent scenarios called ``neutral drift'' in evolutionary theory, which is the case in which coalescence takes the longest.
When some members of the population are more likely to be selected than others (and $N$ is fixed), we expect to see ``selective sweeps'' in which those members of the population outcompete others.
A selective sweep will lead to rapid coalescence.
Thus, in general, in the presence of non-random selection, we expect $T_c$ to be much smaller than we would expect under drift.

Would we ever expect $T_c$ to be larger than under drift? 
Only if lineages coexist for longer than we would expect by chance.
Such a regime can theoretically occur when there is selection for stable coexistence or diversity maintenance.
However, such a regime also involves introducing non-random selection.
Thus, in practice, while these regimes maintain multiple deep branches, there is often frequent coalesence within those branches.
Theoretically, though, extreme pressure for diversity could cause $E(|T|)$ to scale faster than $\matchcal{O}(N)$.

% TODO:
% - Address population size
% - Asexual vs. sexual


\subsection{Comparison between perfect tracking and hereditary stratigraphy}

% Considerations:
% - Distributed vs. not
% - Accuracy
% - Memory fluctuations