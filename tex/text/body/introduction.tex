\section{Introduction} \label{sec:introduction}

Absent indefinite storage capacity, any piece of incoming streaming data must eventually be either evicted or digested if space is to be made available for new input \citep{gaber2005mining}.
This constraint is a crucial consideration in algorithm design for data streams, scenarios involving read-once inputs available only in a strictly ordered sequence.
Such streams' ordering may be dictated by inherently real-time processes (e.g., sensor readings) or retrieval limitations of storage media (e.g., a tape archive) \citep{henzinger1998computing}.
The data streaming model assumes input greatly exceeds memory capacity, with many analyses simply treating streams as unbounded \citep{jiang2006research}.

Data streaming scenarios pervade domains across science and industry \citep{aggarwal2009data,akidau2015dataflow}.
Commercial application areas include sensor networks \citep{elnahrawy2003research}, big-data analytics \citep{he2010comet}, real-time network traffic analysis \citep{johnson2005streams,muthukrishnan2005data}, systems administration \citep{fischer2012real}, and financial analytics for fraud prevention and algorithmic trading \citep{rajeshwari2016real,agarwal2009faster}.
Notable scientific applications arise in environmental/climate monitoring \citep{hill2009real} and astronomy \citep{graham2012data}.
Purely-programmatic computation can also behave as a data stream --- iterative simulation processes traverse vast expanses of ephemeral intermediate state that must be traced to verify simulation dynamics and assess simulation outcomes \citep{abdulla2004simulation,schutzel2014stream}.

Indeed, this broad utility has begat an extensive corpus of data stream algorithms.
Common objectives include rolling summary statistic calculations \citep{lin2004continuously}, on-the-fly data clustering \citep{silva2013data}, live anomaly detection \citep{cai2004maids}, and rolling event frequency estimation \citep{manku2002approximate}.
Data stream algorithms typically draw on one or more of three key stratagems: (1) rolling mechanisms, which restrict consideration to a FIFO tranche of recent data, (2) accumulation, which successively folds data into a summary statistic (e.g., sum, count, etc.) where data is repeatedly applied to a fixed amount of memory or resources, and (3) binning, which consolidates data within time interval bins to create a coarsened record.

Here, we focus on the third stratagem, binning.
Specifically, we develop efficient procedures to maintain temporally-representative subsamples of a data stream on a rolling basis.
That is, to read sequential observations from a data stream on an ongoing basis and sequence their disposal to maintain a record of data stream observations.

We term the rolling management of samples subsetted from a data stream as ``stream curation.''
Proposed algorithms span several possible requirements for two curatorial properties: (1) ``order of growth'' --- how curated collection size should grow in proportion to stream depth and (2) ``gap size bounds'' --- how retained samples should be spaced across stream history.
These considerations arise in various combinations across existing work \citep{aggarwal2003framework,han2005stream}, reviewed in detail later on;
here, we systematize these curatorial properties and contribute novel curatorial policy implementations distinguished by efficiency.
Each contributed policy includes indexing schemes that simultaneously support both efficient update operations and efficient storage of retained stream values in a flat array, requiring only $O(1)$ storage overhead --- a single counter value.

Although we do not treat it directly here, the original motivating application for contributed stream curation algorithms is ``hereditary stratigraphy,'' a recently-developed technique for distributed tracking of copy trees among replicating digital artifacts \citep{moreno2022hereditary}.
Applications of such tracking include phylogenetic analysis of highly-distributed genetic algorithms and evolutionary simulations, and provenance analysis of decentralized social network content, peer-to-peer file sharing, and computer viruses.
A brief description of hereditary stratigraphy is instructive to the timbre of algorithms contributed here.

\subsection{Hereditary Stratigraphy}

In order to reconstruct histories of relatedness, hereditary stratigraphy annotates replicating artifacts with a record of checkpoint fingerprints that grows by accretion with each replication event.
Comparing two artifacts' checkpoint records tells the extent of their common ancestry, as annotations will share common fingerprints up through the time of their last common ancestor and then differ.

Considering generational fingerprint records as a data stream, hereditary stratigraphy applies binning techniques to manage fingerprint accretion --- paring down retained fingerprints while maintaining checkpoints spaced across generations back to the progenitor artifact.
In the context of hereditary stratigraphy, stream curation decides how annotation size scales with generations elapsed by controlling how many retained strata accumulate.
Stream curation decisions directly influence capability for ancestry inference, because the onset of lineage divergence can only be discerned where fingerprints are retained.
Requirements on space usage and inferential power differ substantially between use cases of hereditary stratigraphy, so flexible support for a variety of record size/inferential power trade-offs is crucial.

Of particular note, however, is hereditary stratigraphy's necessity for compact representation of fingerprint records.
Because reduced fingerprint size allows more fingerprints to be retained, typical use will take fingerprints as individual bits, or possibly bytes (to avoid addressability complications).
In this context, representational overhead incurred, e.g., by explicitly storing fingerprints' individual stream sequence indices, can easily bloat annotations' footprint severalfold.
For some use cases, annotated artifacts will number millions or higher, so annotation inefficiency may substantially burden memory, storage, and network bandwidth (i.e., serialized artifact-annotation exchange).

Here, however, we present these algorithmic foundations developed for hereditary stratigraphy in the more generalized frame of data stream processing.
We describe a suite of indexing schemes for stream curation that support (1) linear, logarithmic, and constant scaling relationships between record size and generations elapsed and (2) both even-time and recency-biased distributions of retained stream items.
Implementations provided for each drop representational overhead for curated stream data to a single counter value.
Presented algorithms are published through the \texttt{hstrat} Python package for hereditary stratigraphy \citep{moreno2022hstrat}, but can be directly accessed through public APIs fully independent of other aspects of hereditary stratigraphy methodology.

To provide further introduction to key concepts behind stream curation, the next sections situate our proposed stream curation procedures within existing data stream literature and consider applications of stream curation data loggers and sensor networks.

\subsection{Hereditary Stratigraphy} \label{sec:hereditary-stratigraphy}

Digital systems excel at copying and distributing information \citep{miller2001taking}.
Provenance among copies of digital artifacts can be represented as a tree of parent-child relationships.
The structure of digital ancestry trees can elucidate topics of significant scientific and societal interest, including cybersecurity, misinformation, and cultural preferences \citep{aslan2020comprehensive,dupuis2019spread,ling2021dissecting}.
Here, we contrast direct ancestry tracking with distributed ancestry tracking, outline hereditary stratigraphy methodology for distributed ancestry tracking, and review applications of distributed ancestry tracking.

\subsubsection{Direct Ancestry Tracking}

Most work analyzing ancestry trees of digital artifacts applies centralized lineage tracking \citep{friggeri2014rumor,cohen1987computer,dolson2023phylotrackpy}.%
\footnote{A notable exception, Libennowell et al. exploit a serendipitous mechanism to reconstruct global dissemination of chain emails --- discussed further below.}
Direct approaches to tracking replicator provenance in digital systems represent this graph structure explicitly, distilling it from the full set of parent-child relationships over the history of a population.
This yields an exact historical account.

Unfortunately, typical lineage tracking is difficult to scale.
At scale, practical limitations preclude complete, permanent records of replication events --- which accumulate linearly with elapsed generations \citep{dolson2023algorithms}.
So, extinct lineages are pruned.
Extinction detection requires either (1) collation of all replication and destruction events at a centralized data store or (2) peer-to-peer transmission of extinction notifications that unwind a lineage's (possibly many-hop) trajectory across host nodes.
Both approaches oblige runtime communication overhead.
Further, under the perfect-tracking paradigm single missing relationships can entirely disjoin knowledge of how large portions of phylogenetic history relate.
This makes direct lineage tracking highly sensitive to data loss and dynamic network topology rearrangements --- key considerations at scale \citep{cappello2014toward,ackley2011pursue}.

\subsubsection{Motivation for Decentralized Ancestry Tracking}

Under an alternate framing, phylogenetic biology almost certainly constitutes the single most extensive investigation of artifact provenance ever conducted.
Biological phylogenetics pursues a vast campaign to stitch together millions of lines of descent by analysis of fossil records, phenotypic traits, and --- particularly --- genetic information \citep{hinchliff2015synthesis,lee2015morphological}.
Successes with natural systems, resoundingly evidence viability of robust phylogenetic analyses over fully-distributed processes.

Inspired by systematic biology, we can consider post-hoc inference from heritable data as an alternative to direct tracking.
At core, biological processes of mutation, recombination, conjugation, transformation, transduction, and gene transfer induce patterns of (sufficiently nonvolatile) variation that can be mined to reveal historical relationships \citep{davis1992populations}.

Indeed, digital systems with imperfect copy processes (e.g., digital evolution), phylogenetic reconstructions can be accomplished solely from pre-existing, functional digital genome contents \citep{moreno2021case}.
However, such algorithms are highly dependent on representational and mutational specifics of individual simulation systems and do not readily generalize across diverse digital evolution systems.
Also, they suffer from many of the complications that plague phylogenetic analyses of biological systems.
And in systems without mutational dynamics (i.e., perfect copies), this is not possible.

An intriguing proposition presents itself: how might a genetic representation be designed to maximize ease and efficacy of phylogenetic reconstruction?
If sufficiently compact, such a representation could be attached as an (entirely ornamental) annotation to facilitate phylogenetic inference over genomes of interest.
This line of reasoning led to the recent development of ``hereditary stratigraphy'' methodology, which associates digital artifacts with heritable annotations specially designed to provide universal system-invariant phylogenetic reconstruction with well-defined, tunable expectations for inference accuracy \citep{moreno2022hereditary}.

\subsubsection{Mechanisms for Decentralized Ancestry Tracking}

The core mechanism of hereditary stratigraphy is creation, and subsequent inheritance, of a new randomized data packet each generation.
These stochastic fingerprints, which we call ``differentia,'' serve as a sort of checkpoint for lineage identity at a particular generation, providing for straightforward inference of lineage histories.
At future time points, extant annotations will share identical differentia for the generations they experienced shared ancestry.
So, the first mismatched fingerprint between two annotations bounds the extent of common ancestry.

To prevent bloat of annotation size in direct proportion to generations elapsed, hereditary stratigraphy prescribes a ``pruning'' process to discard   differentia on the fly as generations elapse.
This pruning, however, introduces uncertainty to inference of ancestry.
The last generation of common ancestry between two lineages can be resolved no finer than the time points associated with retained strata.
In the context of hereditary stratigraphy, we refer to rules for downsampling as a ``stratum retention'' algorithm.
Stratum retention must decide how many records to discard, but also how remaining records distribute over past time.
Discussion considers tuning stratum retention trade-offs elsewhere as an instance of the more general ``stream curation'' problem --- introduced separately below in Section \ref{sec:streaming-curation} and presented in detail in Section \section{sec:annotation-algorithms}:

Procedures for tree synthesis from a population of stratigraph annotations follow from the same principle as pairwise annotation comparison, but requires some further elaboration.
Section \section{sec:reconstruction-algorithm} provides a trie-based algorithm for this task.

\subsubsection{Distributed Ancestry Trees in Digital Evolution}

Digital evolution refers to algorithms that combine replication, mutation, and selection to instantiate an evolutionary process.
The field sits at the intersection of evolutionary biology, viz. simulation experiments, and machine learning, viz. heuristic optimization.
Within digital evolution, complications of distributed tracking have historically restricted most phylogenetic analyses to single-process or centralized leader/follower simulations; this lacking motivated original development of hereditary stratigraphy.

Computational scale greatly impacts digital evolution and, more broadly, artificial life \citep{ackley2014indefinitely} --- particularly with respect to open-ended evolution, the question of how (and if) closed systems can indefinitely yield new complexity, novelty, and adaptation --- is inexorably intertwined with computational scale, and has been implicated as meaningfully compute-constrained in at least some systems \citep{taylor2016open,channon2019maximum}.
It is not an unreasonable possibility that orders-of-magnitude changes in computational scale of digital evolution could induce qualitative improvements \citep{moreno2022engineering}, analogously to the renaissance of deep learning with the advent of GPGPU computing \citep{krizhevsky2012imagenet}.

% ELD: TODO: consider cutting this paragraph
Indeed, digital evolution work commonly marshals substantial parallel and distributed computing, already  --- in some cases reaching petaflop scale \citep{gilbert2015artificial}.
Methods range from entirely independent evolutionary replicates across compute jobs \citep{dolson2017spatial, hornby2006automated}, to data-parallel fitness evaluation of single individuals over independent test cases using hardware accelerators \citep{harding2007fast_springer, langdon2019continuous}, to application of primary-subordinate/controller-responder parallelism to delegate costly fitness evaluations of single individuals \citep{cantu2001master,miikkulainen2019evolving}.

Fully decentralized, highly-distributed approaches include as island models \citep{bennett1999building,schulte2010genetic} and mesh-computing approaches prioritizing dynamic interactions \citep{ray1995proposal,ackley2018digital,moreno2021conduit} have also been explored.
However, a major barrier to effective applications of the fully-distributed paradigm has been difficulty of observing system state --- including phylogenetic history, which can provide valuable insight into the mode and tempo of evolution \citep{dolson2020interpreting}.
In the absence of a global population-level view, the course of evolution becomes much more challenging to study --- undercutting applications of these systems as an experimental platform.

\subsubsection{Further Applications of Distributed Ancestry Trees}

Replication of digital artifacts pervades computing, and interest in understanding structure of copy trees extends far beyond digital evolution.
Indeed, some research has reported on the routes through which digital image misinformation and computer viruses spread \citep{friggeri2014rumor,cohen1987computer}.
Such studies generally rely on centralized tracking, which is not possible in many circumstances.

Where possible, decentralized copy tracking has proven an informative asset.
Work by Libennowell et al. on chain email serves as a notable, and rare, example.
These researchers applied \textit{post-hoc} methods to reconstruct estimated phylogenies of the propagation of two chain mail messages.
These phylogenies then served as a reference to tune agent-based models, ultimately yielding a remarkable elucidation of email user behavior and underlying social dynamics.
Interestingly, this study's reconstructions were solely enabled by a special peculiarity of the two sampled messages: they were email petitions.
Thus, users would append their name to the list before forwarding it on --- a mechanism strikingly similar In broad strokes to hereditary stratigraphy.

Hereditary stratigraphic techniques make possible new visibility into hereto cloistered processes through which digital artifacts spread.
Take, for example, peer-to-peer and federated social network, which have recently enjoyed substantial upticks in popularity \citep{la2021understanding}.
Within social networks, the structure of share chains is key to understanding, detecting, and mitigating misinformation \citep{kucharski2016study,raponi2022fake} and predatory viral marketing (e.g., spamming and phishing attacks) \citep{guidi2018managing}.
However, existing work on share chain structure typically relies on the capacity for direct ancestry tracking afforded by monolithic platforms (e.g., Facebook, Twitter, etc.).
Hereditary stratigraphy could enable robust, semi-anonymized extraction of share chain diagnostics within decentralized social networks.

Substantial limitations on uses cases for hereditary stratigraphy should be noted, however.
Data collection predicates influence over the artifact copy process, necessary to ensure performance of stratigraph annotation updates.
Additionally, the methodology is susceptible to defacement of annotation data by antagonistic actors.

\subsection{Stream Curation} \label{sec:streaming-curation}

Under an iterative model, time operates something like a first-in, nothing-out queue --- successive time steps simply pile on ad infinitum.
As time accumulates, an elapsed time step recedes ever deeper.
A discrete event's time point does not change, but its relation to the present does.  % cite xckd 1477?
This inevitability is crux to the stream curation problem, which we establish to describe rolling maintenance of a temporally representative cross-section of sequenced observations.
For the sake of generality, we employ this framing to discuss stratum retention strategies for hereditary stratigraphy.
stream curation relates directly to the concept of binning in data streams, which we discuss further below.

Stream curation algorithms must answer how many observations should be kept at any point in time, but also how retained observations should be spaced out over past time.
Appropriate choices vary by use case, and no stream curation policy can meet requirements of all use cases.
For this reason, we consider a spectrum of balances between two factors: size limitation, i.e., how many observations may be retained, and resolution guarantees, i.e., maximum gap sizes.
For each space-resolution stipulation profile, we provide an implementation algorithm meeting criteria of computational reducibility and self-consistency, defined below.
After discussing policy stipulation criteria and policy implementation for stream curation, we close with connections to existing work and potential applications.

\subsubsection{Stream Curation Policy Stipulation}

Functional requirements for stream curation policy delineate collection size, i.e., total observations retained, and gap size, i.e., length of time windows between retained observations.

For the former, we specify bounds on retained observation count as fixed or a function of time elapsed.
We refer asymptotic bounds on the scaling relationship between size and time as  the ``size order of growth.''
We also define hard bounds, referred to as ``size cap,'' vis-a-vis practical considerations, e.g., fixed size memory allocation for curated collection storage.

Bounds on spacing between retained observations, ``resolution guarantee,'' may depend on both the total number of generations elapsed but the historical depth of a particular time point in the stratigraphic record.
Again, we treat both asymptotic and hard bounds.
Considering historical depth provides for configurability of skew in observation density.
Strata from evenly-spaced time points may retain uniform detail over the entire range of elapsed time points.
Alternately, retention allocations may space successive retained observations proportionately to historical depth.
Such an approach biases observational detail to recent time.
Our surveyed policy stipulations include both even and recency-proportional resolutions.

In the context of hereditary stratigraphy, recency-proportional resolution is typically preferable.
Coalescent theory predicts a tendency for evolution-like processes to produce phylogenies with many recent branches and progressively fewer ancient branches \citep{nordborgCoalescentTheory2019, berestyckiRecentProgressCoalescent2009}.
Thus, fine inferential detail over recent time points is usually more informative to phylogenetic reconstruction than detail over more ancient time points.
Indeed, experiments reconstructing known lineages have shown that recency-skewing retention provides better quality reconstructions \citep{moreno2022hereditary}.

Note that size bounds and resolution guarantees must hold across all time points, necessitated by use cases where observation collections will see sustained use over time or the endpoint for an observation collection is indeterminate (e.g., computations with a real-time termination condition).
This factor introduces design subtlety: as generations elapse and observations become more ancient, resolution guarantees may shift.
In those cases, cohorts of retained strata must, in dwindling, gracefully morph through a constrained series of retention patterns.

\subsubsection{Stream Curation Policy Algorithms}

A stream curation policy algorithm produces a sequence of observation retention sets, one for each time point.
First and foremost, each of an algorithm's retention sets should satisfy stipulated requirements on collection size and gap size.

To be viable, each retention set's complement (i.e., discarded observations) must superset the complement of all preceding retention sets.
Otherwise, a discarded observation would be called for inclusion.
Put another way, policy algorithms must ensure no retained observations have been discarded at a prior time point.
We call this property self-consistency.

We impose a further nuts-and-bolts requirement on algorithm implementation: computational reducibility, meaning that time points retained at every position in the observation sequence must be directly enumerable.
This capability enables observations' time point to be deduced positionally, so observation time points may be omitted from an observation set's representation.
In the context of hereditary stratigraphy, several-fold space savings may result (depending on differentia bit width).

Since the austere early days of computing, typical hardware has trended away from resource scarcity, making lean memory use less salient \citep{kushida2015cloud}.
Nevertheless, memory efficiency remains crucial in certain contexts where hardware trends have stagnated or even regressed memory capacity,

High-performance computing (HPC) expects to continue emphasis of scale-out with lean processing cores rather than boosting capacities of individual processing components \citep{sutter2005free,morgenstern2021unparalleled}.
The Cerebras Wafer-Scale Engine epitomizes this trend, packaging an astounding 850,000 computing elements onto a single die.
Individual core, however, host just 48kb of memory and operate through a exclusively-local mesh communication model \citep{cerebras2021wafer,lauterbach2021path}.

As with HPC, component economization and minaturization influenced the Internet of Things (IoT) revolution \citep{rfc7228,ojo2018review}, a march potentially culminating in ``smart dust'' \citep{warneke2001smart} of downscale, low-end hardware.
The Michigan Micro Mote platform for instance, provisions a mere 3kb of retentive memory within its cubic millimeter form factor \citep{lee2012modular}.
More recent work has explored devices tucked within the form factor of dandelion parachutes \citep{iyer2022wind}.
The chipset is yet more austere, provisioning 2 kilobytes of volatile flash memory --- and a mere 128 bytes of retentive memory \citep{microchip2014atiny20}.
As engineers continue to plumb the extremities of technical feasibility, bare-bones computing modalities will persist, and applications in wireless sensor networks will necessitate lightweight data stream algorithms.

\subsubsection{Existing Work}

Stream curation relates to existing binning procedures that group together and consolidate contiguous subsections of a data stream.

The fixed-resolution policy algorithm presented here is simple downsampling via decimation \citep[p. 31]{crochiere1983multirate}
Our depth-proportional resolution (Section \ref{sec:depth-proportional-resolution-algo}) and recency-proportional resolution (Section \ref{sec:recency-proportional-resolution-algo}) algorithms share close structural similarity with the online equi-segmented and vari-segmented schemes proposed in
\citep{zhao2005generalized}.
The depth-proportional resolution structure additionally appears as``pyramidal'' and ``tilted'' time window schemes \citep{aggarwal2003framework,han2005stream}.
However, provided implementations all unfold through stateful iteration, with representational overhead for each stored value (viz. segment length values).
To our knowledge, stateless enumerations of retained set composition are original to this work.
We are also not aware of existing equivalents or near-equivalents of presented geometric sequence $n$th root or curbed recency-proportional resolution policy algorithms (Sections \ref{sec:geom-seq-nth-root-algo} and \ref{sec:curbed-recency-proportional-resolution-algo}).

Looser relations include work on ``amnesic approximation'' a generalized scheme to incrementally down sample a data stream pursuant to a user-defined elapsed-time-to-value function.

\subsubsection{Applications of Stream Curation}

The correspondences between stream curation and more general binning on data streams open avenues apply stratum curation policy algorithms in various data stream scenarios.

Perhaps most plainly, the down sampling task faced by hereditary stratigraph annotations parallel those faced by data logger devices.
Unattended loggers manage incoming observation streams on a potentially indefinite or indeterminate basis.
Devices incorporated into wireless sensor networks also experience irregular device uplink schedules.
The ``mobile sink'' paradigm \citep{jain2022survey}, for example relies on network base station(s) physically traverse the coverage area and transact with nearby sensor nodes.
Sporadic patrol schedules can thus introduce uncertainty in data transfer schedules.

Reported strategies to handle stream data in excess of storage capacity on loggers include rolling full retention of most recent data within available buffer space \citep{fincham1995use} and dismissal of incoming data after storage reaches capacity \citep{saunders1989portable,mahzan2017design},
Strategies to maintain a cross-sectional time sample do not appear to be widely exploited, although there has been some work to extend the record capacity of data loggers through application-specific online compression algorithms \citep{hadiatna2016design}.
Storage-limited platforms might suit proposed memory-efficient stream curation procedures.



Remaining exposition in this paper is structured as follows:
\begin{itemize}
% \item Section \ref{sec:methods} covers preliminaries and glossarizes key terminology,
\item Section \ref{sec:annotation-algorithms} surveys a suite of streaming curation algorithms, introducing intuition, presenting the formal definition, proving self-consistent stratum discard sequencing, and demonstrating resolution and collection size properties.
% TODO make sure this gets into the actual text
\item Section \ref{sec:reconstruction-algorithm} presents a recently-developed algorithm for full-tree reconstruction from hereditary stratigraphic annotation data and analyzes its runtime characteristics.
% \item Section \ref{sec:perfect-tracking-algorithm} supplies formal presentation of the alternate perfect phylogenetic tracking algorithm and analysis of its runtime characteristics then commentates on which situations better suit perfect tracking over hereditary stratigraphy, and vice versa.
% TODO do we need a results and discussion section???
% \item Section \ref{sec:results-and-discussion}
\item Section \ref{sec:conclusion} reflects on broader implications and future work, and
\item we include a Glossary of terminology related to hereditary stratigraphy, streaming curation, and phylogenetics in the Appendix.
\end{itemize}
