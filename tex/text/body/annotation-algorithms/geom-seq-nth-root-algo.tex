\subsection{Geometric Sequence $n$th Root (GSNR) Policy Algorithm}
\label{sec:geom-seq-nth-root-algo}

% The geometric sequence $n$th root (GSNR) policy algorithm tackles recency-proportional uncertainty objectives within a fixed memory capacity constraint.  is that ok??????? yeah that looks good!! fixed-size memory? among observations? i like it!
The geometric sequence $n$th root (GSNR) policy algorithm arranges recency-proportional gap sizes among a capped-size set of retained observations.
% Although recency-proportional gap size will not be bounded below a fixed threshold in this context, GSNR seeks to minimize it to the extent possible.
Although recency-proportional gap size will not be bounded to a fixed threshold in this context, GSNR seeks to minimize worst relative gap size as much as possible.

Recall that the recency-proportional policy algorithm (RPR) exhibits logarithmic growth in extant record size with respect to record depth $n$.
When an increased order of magnitude depth is reached, additional observations must be retained under the RPR algorithm.
For $a = \log_b(n)$, $a$ is proportional to extant record size.
Equivalently, $b^a = n$.
Under RPR, growth in extant record size can be roughly conceptualized as related to insufficiency of the base $b$ to reach $n$ within $a$ multiplicative steps.
Growth in $a$ --- i.e., additional multiplication by $b$ --- can be thought of in terms of adding a level of structural hierarchy within the layout of retained observations.
As $n$ increases, additional levels of structural hierarchy become necessary.
These additional hierarchical levels increase extant record size.
% Put another way, progression of the policy algorithm accommodates additional appending an additional level of hierarchy.
% ELD: I feel like I was supposed to come away from this paragraph understanding what problem GSNR is trying to solve, but I don't
% @ELD 2024: this comment applies more to the paragraph before this one now, but I think we're still not doing a great job of motivating GSNR

In order to prevent such unbounded growth, the GSNR policy algorithm fixes the number of hierarchical levels $a$ and accommodates additional record depth by adjusting the multiplicative factor $b$.
This scheme can be imagined enforcing an arrangement of $a$ exponentially-spaced target points along the historical record.
% ELD: I know we don't have time or space, but I think a little picture of this would help a lot
As time elapses, the quantity of target points remains constant.
The target points shift to fill $n$ by increasing their exponential spacing factor $b$.
The necessary magnitude of $b$ works out as $b = n^{1/a}$.
Target ages therefore correspond to $n^{0/a}, n^{1/a}, \ldots, n^{a/a}$.
This geometrically-spaced target point sequence eponymizes the GSNR policy algorithm.
Converting observation age to absolute time point, targets span $n - n^{0/a}, n - n^{1/a}, \ldots, n - n^{a/a}$.

% ELD: It might be worth trying to sign-post earlier that this algorithm is the way it is because that happens to work well, and that its not something that should be obvious to the audience

Now, attention turns to exploiting the $n$th root geometric targets to define a retention policy.
We will break the problem down to consideration of one individual target point $n - n^{x/a}$.
Under the constraint of $\mathcal{O}(1)$ total space for curated observations, we can only curate a fixed number of observations per target point.
We will seek to curate a fixed size collection of retained observations to bound gap size past the target point below $n^{x/a}$.

By nature of definition, target point times advance monotonically.
As a consequence, a retained observation can remain behind a target point indefinitely.
We will incorporate such coverage into our design --- let's call such a point behind the target the ``backstop'' $\beta$.
% Because the target point shifts with record depth, but retained time points are immutable, no retained backstop time point can exactly track the target point.
% ELD: Maybe "use" instead of "reach for"? Also maybe power of 2 instead of binary power? It took me a while to parse what a binary power is
% Instead, we will follow the vein of the RPR policy algorithm and the depth-proportional resolution policy algorithm and reach for a binary power.

We will use a power of 2 trick to maintain backstop coverage.
To begin, let us take the binary floor of half $n^{x/a}$,
% ELD: I'd either make this sentence have more intuition or cut it.
% TODO we should probably define binary floor in preliminaries and refer there
% addendum: we no longer have preliminaries
\begin{align*}
  \kappa(n)
  &=
  2^{\lfloor \log_{2}(n^{x/a}/2) \rfloor}.
\end{align*}
We will retain recent time points that are multiples of this value $\kappa$.
Note that $n$ strictly increasing implies $\kappa(n)$ monotonically increasing.

Let's define a floor $B$ to help place our backstop $\beta$,
\begin{align*}
  B(n)
  &=
  \max \left(
    n - \left\lceil  \frac{3n^{x/a}}{2} \right\rceil,
    0
  \right)
\end{align*}
By design, $B$ precedes target point $n - n^{x/a}$.
Again, with $x < a$, $n$ strictly increasing implies $B(n)$ monotonically increasing.

Rounding $B$ up to the next time point aligned to cadence $\kappa$ gives our backstop time point $\beta$,
\begin{align*}
  \beta(n)
  &=
  B(n) + \big(-B(n) \bmod \kappa(n)\big).
\end{align*}
% ELD: there's a weird gap in this equation, but it's probably not worth worrying about
% MAM: latex mod formatting is always weird and ugly
% SRP: \bmod is quirky and cute :3
It can be shown that $\beta(n) \leq n^{x/a}$.
Because $B(n)$ is monotonically increasing, $\beta(n)$ is as well.

We will retain the time point set $S_x$ comprising multiples of $\kappa$ at or after the backstop $\beta$,
\begin{align*}
  S_x(n) = \{\, t \mid \beta(n) \leq t < n \text{ and } t \bmod \kappa(n) = 0 \,\}.
\end{align*}

Why does this construction for target $n^{x/a}$ satisfy our $\mathcal{O}(1)$ space complexity constraint?
It can be shown that $|S_x(n)| \leq 6$.
This stems from cadence $\kappa$ as the binary floor of half $n^{x/a}$ (at most a quartering reduction) and backstop $\beta$ set at most $3/2 \times n^{x/a}$ time points back.

% Because cadence $\kappa$ is the binary floor of half $n^{x/a}$, we have $n^{x/a} \leq 4\kappa$.
% Recall that $\beta \geq n - 3 n^{x/a} / 2$.
% Remark that
% \begin{align*}
% |S_x(n)|
% = \lfloor \frac{n - \beta}{\kappa} \rfloor.
% \end{align*}

% Relaxing the integer floor,
% \begin{align*}
% |S_x(n)|
% \leq \frac{n - \beta}{\kappa}.
% \end{align*}

% Substituting inequalities,
% \begin{align*}
% |S_x(n)|
% &\leq 4\frac{n - \beta}{4\kappa}\\
% &\leq 4\frac{n - \beta}{n^{x/a}}
% \end{align*}

% and then

% \begin{align*}
% |S_x(n)|
% \leq 4\frac{n - \Big(n - 3 n^{x/a} / 2\Big)}{n^{x/a}} \\
% \leq 4\frac{3 n^{x/a} / 2}{n^{x/a}} \\
% \leq 4 \times \frac{3}{2} \\
% \leq 6.
% \end{align*}
% Theorem \ref{thm:geom-seq-nth-root-algo-space-complexity} details this point.

Why does this curated set for target $n - n^{x/a}$ satisfy our gap size bound $n^{x/a}$?
Because cadence $\kappa \leq \frac{n^{x/a}}{2}$, gap size satisfies the bound.
Because backstop $\beta \leq n - n ^{x/a}$, the target time point is covered within the cadenced range.
% Theorem \ref{thm:geom-seq-nth-root-algo-uncertainty-bound} details this point.

We bring curated sets for each target point together in a set union to produce the overall GSNR retained set $R$,
\begin{align}
  \mathsf{GSNR\_kept}(\colorT)
  &=
  \bigcup_{i=0}^{a} S_i(n).
\label{eqn:gsrn_kept}
\end{align}
Note that under this construction, policy algorithm self-consistency and extant record size bounds follow from those shown for constructions for individual targets $n^{x/a}$.
We demonstrate self-consistency over elapsed time $\colorT$ in Suplementary Theorem \ref{thm:geom-seq-nth-root-algo-self-consistency}.
\footnote{
A strict upper bound of $6a + 2$ for extant record size can be calculated, although we do not demonstrate it here.
}
Time points that should be dropped to enact the GSNR policy algorithm follow from set subtraction between $\mathsf{GSNR\_kept}(n)$ and $\mathsf{GSNR\_kept}(n+1)$.

We have discussed resolution guarantees for individual target point constructions, but what resolution guarantee is afforded overall for an arbitrary time point $t$ with age $g = n - t$?
For such a time point $t$, the tightest resolution guarantee is that of the next older target point.
Taking
\begin{align*}
\alpha = n^{1/a}
\end{align*}
the age of the next older target time point will be
\begin{align*}
\alpha^{ \lceil \log_{\alpha} g \rceil },
\end{align*}

By the target time point resolution guarantee we established earlier, the gap size provided at this target time point is bounded by its age.

Observe that, at most, the next older target time point age will be a factor of $\alpha = n^{1/a}$ greater than $g$.
So, the worst case absolute provided gap size is
\begin{align*}
\alpha \times g = n^{1/a} \times g.
\end{align*}

Worst case recency-proportional gap size is therefore $n^{1/a}$.

Figure \ref{fig:retention-policies} includes a time-lapse of the extant record under the GSNR policy algorithm.
The distinguishing feature of the GSNR policy algorithm is it keeps recency-proportional gap sizes past the point where RPR policy would overflow a given record size bound.
% ELD: are we saying "effort" because its an in practice benefit but not a provable asymptotic one?
% MAM: trying to mean that it's asymptotic but not perfect
This lends it to very-long duration applications.
% However, before very-deep record depths, we have observed that recency-proportional resolution tends to better distribute equivalent numbers of retained observations.

% % TODO: missing proof
\begin{theorem}{Geometric Sequence $n$th Root Self-Consistency}
\label{thm:geom-seq-nth-root-algo-self-consistency}

\end{theorem}

\begin{proof}
\label{prf:geom-seq-nth-root-algo-self-consistency}

\end{proof}

% % TODO: missing proof
\begin{theorem}{Geometric Sequence $n$th Root Space Complexity}
\label{thm:geom-seq-nth-root-algo-space-complexity}
The \gls{extant record size} of the Geometric Sequence $n$th Root Policy Algorithm grows with order $\mathcal{\theta}{(k)}.$

\end{theorem}

\begin{proof}
\label{prf:geom-seq-nth-root-algo-space-complexity}



\end{proof}

% \begin{theorem}{Geometric Sequence $n$th Root Uncertainty Bound}
\label{thm:geom-seq-nth-root-algo-uncertainty-bound}

\end{theorem}

\begin{proof}
\label{prf:geom-seq-nth-root-algo-uncertainty-bound}

\end{proof}


% % TODO: missing pseudocode
\begin{algorithm}
\caption{Geometric Sequence $n$th Root Stratum Enumeration}
\label{alg:geom-seq-nth-root-gen-drop-ranks}
\begin{algorithmic}[1]
\State{Define $P:=T:=\{ \{1\},\ldots,\{d\}$\}}
\WHILE{$\#P > 1$}
\State{Find an optimal partition tree $T_{C^\prime}$ }
\State{Update $P := (P{\setminus} C^\prime) \cup \{ \bigcup_{t\in C^\prime} t \}$}
\State{Update $T := T \cup \{ \bigcup_{t\in\tau} t : \tau\in T_{C^\prime}{\setminus}
\mathcal{L}(T_{C^\prime})\}$}
\ENDWHILE
\Return $T$
\end{algorithmic}
\end{algorithm}

% % TODO: missing pseudocode
\begin{algorithm}
\caption{Geometric Sequence $n$th Root Stratum Retention Predicate}
\label{alg:geom-seq-nth-root-algo-pred-keep-rank}
% STUB ALGORITHM
\end{algorithm}

\begin{algorithm}
\caption{Geometric Sequence $n$th Root Stratum Enumeration}
\label{alg:geom-seq-nth-root-enum-retained-ranks}
\begin{algorithmic}
    \Require{ $\texttt{n}$ -- the number of strata deposited }
    \Require{ $\texttt{p}$ -- the interspersal  }
    \Require{ $\texttt{d}$ -- the degree }
    \Ensure{ set of retained strata }
\end{algorithmic}

\begin{algorithmic}[1]
    \State $\texttt{set} \gets {0, n - 1}$

    \For{$p = 1$ \textbf{to} $\texttt{d} + 1$}
        \State $\texttt{r} \gets n^{\frac{p}{d}}$

        \State $\texttt{k} \gets \text{largest integral power of two} \le \max \{\frac{r}{p}, 1 \}$ % retained_ranks_sep

        \State $\texttt{c} \gets \max \{n - \ceil{r + \frac{r}{p}}, 0\}$
        \State $\texttt{a} \gets \texttt{c} - (\texttt{c} \bmod -\texttt{k})$

        \State $\texttt{set} \gets \texttt{set} \cup \{\texttt{a}, \texttt{a} + \texttt{k}, \texttt{a} + 2\texttt{k}, {a} + 3\texttt{k}, \dots, n\}$
    \EndFor
\end{algorithmic}
\end{algorithm}

