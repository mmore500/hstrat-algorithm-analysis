\subsection{Recency-proportional Resolution (RPR) Policy Algorithm}
\label{sec:recency-proportional-resolution-algo}

This stream curation algorithm's properties fall between the properties of the fixed resolution (FR) and depth-proportional resolution (DPR) policy algorithms, covered in the immediately preceding sections.

Recall that the DPR policy algorithm's gap widths grow in linear proportion to record depth.
In contrast, the fixed resolution algorithm's gap width remains constant below a specified bound across record depths.

The recency-proportional resolution (RPR) policy algorithm bounds gap width to a linear factor of layer age (i.e., time steps back from the newest layer).
Here, layer refers to the $m$th observation ingressed from the underlying data stream being curated.

Suppose $n$ data stream observations have elapsed.
% set k as age and explain that then replace n - m
% annotate-equations (latex math annotation?)
Then, for a user-specified constant $r$, no gap width for layer $m$ will exceed size
\begin{align}
  \left\lfloor \frac{n - m}{r} \right\rfloor.
  \label{eqn:rpr-gap}
\end{align}
Resolution at each layer widens linearly with record depth.
Consequently, resolution widens in linear proportion to layer age.
Resolution for any given layer age, however, remains constant for all record depths.

The FR and DPR policy algorithms exhibit $O(r)$ and $O(rn)$ extant record orders of growth, respectively.
We will show extant record order of growth as $O(r\log{n})$ under the recency-proportional resolution policy algorithm.

Algorithm \ref{alg:recency-proportional-algo-gen-drop-ranks} enumerates time points of dropped observations under the RPR policy algorithm.
Figure \ref{fig:retention-policies} includes a time-lapse of the extant record under this policy algorithm.

The extant record is determined iteratively, beginning at observation time zero --- which is always retained.
Per Equation \ref{eqn:rpr-gap}, gap width to the next retained observation can be at most $\lfloor n/r \rfloor$ sites, where $n$ is record depth.
Although retaining the observation at time $\lfloor n/r \rfloor$ would satisfy policy resolution guarantees, a slight complication is necessary to ensure self-consistency.
A fuller rationale will follow, but in short, gap width is floored to the next lower power of two,
\begin{align*}
  2^{\lfloor \log_{2}\left(\frac{n}{r}\right) \rfloor}.
\end{align*}
The next iteration repeats the procedure from the newly retained observation time instead of from time zero.
Iteration continues until reaching the newest observation.

The set of observations to eliminate can be calculated from set subtraction between enumerations of the historical record at time points $t-1$ and $t$.
So, update time complexity follows from extant record enumeration time complexity, which turns out to be $O(\log n)$.
We provide a tested, but unproven, constant-time pruning enumeration implementation in the \texttt{hstrat} library, but will not cover it here. % mention why?
The extant record order of growth of $O(\log n)$ also follows from the record enumeration algorithm, as detailed in Theorem \ref{thm:recency-proportional-resolution-algo-space-complexity}.

% TODO figure?

Why does flooring step sizes to a binary power ensure self-consistency?
Let us begin by noting properties applicable to all layers $l$,
\begin{enumerate}
\item gap width provided at retained layer $l$ increases monotonically as record depth grows,
\item the retained observation preceding or at $l$ has observation time at an even multiple of surrounding gap widths, and
\item all observations at time points that are multiples of gap width past $l$ up to the newest observation are retained.
\end{enumerate}
Observe that gap width decreases monotonically with decreasing layer age (i.e., increasing layer recency).

Properties 2 and 3 occur as a result of stacking monotonically-decreasing powers of two.
Subsequent smaller powers of two tile evenly to all multiples of a larger power of 2, giving property 3.
Conversely, preceding larger powers of 2 can be evenly divided by succeeding smaller powers of 2, ensuring that the edges of smaller powers of 2 gaps occur at even multiples of their gap width, giving property 2.

Under the binary flooring procedure, when gap size increases at a layer it will double (or quadruple, octuple, etc.).
Availability of the new gap endpoint after a gap size increase is guaranteed from the tiling properties due to that endpoint being an even multiple of original step size.
% Theorem \ref{thm:recency-proportional-resolution-algo-self-consistency} uses inductive proof-by-contradiction to show this self-consistency.

% @ELD could cut this paragraph @MAM putting in footnote
The RPR policy algorithm provides stable relative accuracy indefinitely.
This makes it particularly attractive in applications to phylogenetic tracking scenarios using hereditary stratigraphy.
To meaningfully describe an ancestry tree with deep branches, information must be retained across all evolutionary time but higher absoute estimation error is typically acceptable in describing more ancient most recent common ancestor (MRCA) events.%
\footnote{%
At comparable annotation sizes, we have found that recency-proportional distribution of gap widths outperforms even gap width distribution in phylogenetic information recovery \citep{moreno2022hereditary}.
Preliminarily, maintaining 3\% relative precision appears sufficient to eliminate most bias from reconstruction error on phylogenetic metrics \citep{moreno2023toward}.
}
%TODO add details about coalescent theory being a good idea for recency-proportional
%@ELD I dont know that we still need to do this, although I guess coalescent theory applies to a lot of processes, so maybe it would be worth invoking in the general case

The RPR policy algorithm's indefinite stability may be particularly useful in scenarios of indefinite or indeterminate record keeping duration.
Although annotation extant record size grows unboundedly, logarithmic memory usage growth is manageable in most practical scenarios.
However, this policy would not suit applications with hard caps on annotation size. %(e.g., fixed memory footprint digital genomes).

\begin{algorithm}
\caption{Recency-proportional Resolution Stratum Discard Generator}
\label{alg:recency-proportional-algo-gen-drop-ranks}
\begin{algorithmic}
    \Require{ $\texttt{n}$ -- the number of strata deposited }
    \Require{ $\texttt{r}$ -- the fixed resolution desired }
    \Ensure{ array of dropped strata }

    \Procedure{NumberToCondemn}{$\texttt{n}, \texttt{r}$}
        \If{$(\texttt{n} \bmod 2 = 1) \lor (\texttt{n} < 2 \cdot \texttt{r} + 1)$}
            \Return $0$
        \Else
            \Return $1 + \Call{NumberToCondemn}{$\texttt{n} / 2$, \texttt{r}}$
        \EndIf
    \EndProcedure

    \State $\texttt{num\_to\_condemn} \gets \Call{NumberToCondemn}{\texttt{n}, \texttt{r}}$
    \State $\texttt{arr} \gets \text{empty array of length num\_to\_condemn}$

    \For{$i = 0$ \textbf{to} $\texttt{num\_to\_condemn} - 1$}
        \State $\texttt{arr} [$i$] \gets \texttt{n} - 2^{i} \cdot (2 \texttt{r} + 1)$
    \EndFor
\end{algorithmic}
\end{algorithm}

% \begin{algorithm}
\caption{Recency-proportional Stratum Retention Predicate}
\label{alg:recency-proportional-resolution-algo-pred-keep-rank}
\begin{algorithmic}[1]
    \STATE $\text{resolution} \gets \text{spec.GetRecencyProportionalResolution()}$
    \STATE $\text{cur\_rank} \gets 0$
    \STATE $last\_rank \gets \texttt{num\_\text{strata}\_\text{deposited}} - 1$
    \WHILE{$last\_rank - \text{cur\_rank} > \text{resolution}$}
        \STATE \textbf{yield} $\text{cur\_rank}$
        \STATE $\text{cur\_rank} \gets \text{cur\_rank} + \texttt{calc\_provided\_uncertainty(} \text{resolution}, \text{last\_rank - cur\_rank} \texttt{)}$
    \ENDWHILE
    \FOR{$i = \max(\text{last\_rank} - \text{resolution}, 0)$ \textbf{to} $\texttt{num\_\text{strata}\_\text{deposited}} - 1$}
        \STATE \textbf{yield} $i$
    \ENDFOR
\end{algorithmic}
\end{algorithm}

% \begin{algorithm}
\caption{Recency Proportional Resolution Stratum Enumeration}
\label{alg:recency-proportional-resolution-algo-enum-retained-ranks}
\begin{algorithmic}[1]
    \Require{ \colorT -- the number of strata deposited }
    \State $\colorTbar \gets 0$
    \While{$\colorTbar < \colorT$}
        \State \textbf{yield} $\colorTbar$
        \State $\colorTbar \gets \colorTbar + \max\Big(\left \lfloor \frac{\colorT  - 1 - \colorTbar}{r + 1} \right \rfloor_{\mathrm{bin}}, \;\; 1\Big)$
    \EndWhile
\end{algorithmic}
\end{algorithm}


% \begin{theorem}{Recency-proportional Algorithm Self-Consistency}
\label{thm:recency-proportional-algo-self-consistency}
Let $\mathsf{RPR\_kept}(n)$ be the set of data points retained at time point $n$, defined per Equation \ref{eqn:dpr_kept}.
Showing self-consistency requires that we demonstrate $\forall m \geq 0, n \in [0 \twodots m)$,
\begin{align*}
\mathsf{RPR\_kept}(n)
\geq
\mathsf{RPR\_kept}(m)
\setminus
\mathsf{future}(n, m).
\end{align*}
where $\mathsf{future}(n, m) = [n \twodots m - 1)$.
\end{theorem}

\begin{proof}
\label{prf:recency-proportional-algo-self-consistency}
Our goal is equivalent to showing that $\forall m \geq 0, n \in [0 \twodots m)$,
\begin{align*}
\mathsf{RPR\_kept}(n) \cup \mathsf{future}(n, m) \geq \mathsf{RPR\_kept}(m).
\end{align*}

Let
\begin{align*}
R_m
&=
\max\Big(\left \lfloor \frac{m  - 1 - \colorTbar}{r + 1} \right \rfloor_{\mathrm{bin}}, \;\; 1\Big)\\
R_n
&=
\max\Big(\left \lfloor \frac{n  - 1 - \colorTbar}{r + 1} \right \rfloor_{\mathrm{bin}}, \;\; 1\Big).
\end{align*}
Note that we have both $R_m, R_n \in 2^{\mathbb{N}}$.
Further, because $n < m$ we have $R_n \leq R_m$.
From these two observations, we may remark that $R_n$ evenly divides $R_m$
\begin{align}
R_m \bmod R_n = 0.
\label{eqn:rmrn0a}
\end{align}

Suppose $\colorTbar \in \mathsf{RPR\_kept}(m)$.
In this case, at least one of the following is true (A) $k = m - 1$ or (B) $k \bmod R_m = 0$.

In the first case, $\colorTbar \stackrel{\checkmark}{\in} \mathsf{future}(n, m)$.
In the second case, we may conclude from Equation \ref{eqn:rmrn0a} that $\colorTbar \bmod R_n = 0$.
So, $\colorTbar \stackrel{\checkmark}{\in} \mathsf{RPR\_kept}(n)$.
\end{proof}

\begin{theorem}{Recency-proportional Resolution Space Complexity}
\label{thm:recency-proportional-resolution-algo-space-complexity}

The \gls{extant record size} of the Recency-proportional Resolution Policy Algorithm grows with order $\mathcal{theta}{(k \log{n})}.$

\end{theorem}

\begin{proof}
\label{prf:recency-proportional-resolution-algo-space-complexity}
As per \ref{sec:extant_record_oog}, we will set out to prove that output array of this policy algorithm has an order of growth of $\mathcal{theta}{(k \log{n})},$ where $k$ is a user-provided resolution and $n$ is the number of depositions.

Algorithm \ref{alg:recency-proportional-algo-gen-drop-ranks} determines the array of strata to be dropped at any given time point.
Observe that whenever $2 \mid n$ at least one stratum will be dropped.
More generally, for any positive integer $i \le log_2{n}$, we have that if $2^i \mid n$ then $i$ strata will be dropped. 
Thus, the number of dropped strata is bounded above by $\sum_{i=1}^{log_2{n}} n = n \log_2{n}.$
As such, the number of retained strata is bound by $n - n \log_2{n} \le \log_2{n}$ for all positive $n.$
Given that no strata are dropped when $\frac{n}{2} - 1 < k,$ we observe that the output array of this policy algorithm is bound above by $\mathcal{O}{(k \log{n})}.$
Via \ref{sec:extant_record_oog}, we can conclude that this bound is actually $\mathcal{theta}{(k \log{n})}.$
\end{proof}

% % TODO: missing proof
\begin{theorem}{Recency-proportional Resolution Uncertainty Bound}
\label{thm:recency-proportional-resolution-algo-uncertainty-bound}

\end{theorem}

\begin{proof}
\label{prf:recency-proportional-resolution-algo-uncertainty-bound}

\end{proof}

