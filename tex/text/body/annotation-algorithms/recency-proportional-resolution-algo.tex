\subsection{Recency-proportional Resolution (RPR) Policy Algorithm}
\label{sec:recency-proportional-resolution-algo}

This stratum retention algorithm's properties fall between the properties of the fixed and depth-proportional resolution policy algorithms, covered in the immediately preceding sections.

Recall that the depth-proportional resolution policy algorithm's gap widths grow in linear proportion to record diuturnity.
In contrast, the fixed resolution algorithm's gap width remains constant below a specified bound across record diuturnities.

The recency-proportional resolution algorithm bounds gap width to a linear factor of layer age (i.e., time steps back from the newest layer).
Suppose $n$ depositions have elapsed.
Then, for a user-specified constant $r$, no resolution for layer $m$ will exceed width
\begin{align}
  \left\lfloor \frac{k}{r} \right\rfloor.
  \label{eqn:rpr-gap}
\end{align}
Resolution at each layer widens linearly with diuturnity.
Consequently, resolution widens in linear proportion to layer age.
Resolution for any given layer age, however, remains constant for all diuturnities.

The fixed and depth-proportional resolution policy algorithms exhibit $O(r)$ and $O(rn)$ residuum orders of growth, respectively
We will show residuum order of growth as $O(r\log{n})$ under the recency-proportional resolution retention algorithm.

Algorithm \ref{alg:recency-proportional-resolution-algo-enum-retained-ranks} enumerates deposition times of retained strata under the recency-proportional resolution retention algorithm.
Figure \ref{fig:retention-policies} shows a time-lapse of the extant record enacted as a result.

The extant record is determined iteratively, beginning at deposition time zero --- which is always retained.
Per Equation \ref{eqn:rpr-gap}, gap width to the next retained stratum can be at most $\lfloor n/r \rfloor$ sites, where $n$ is diuturnity.
Although selecting to retain the $\lfloor n/r \rfloor$ deposition time stratum would satisfy policy resolution guarantees, a slight complication is necessary to ensure that retained strata have been consistently retained since deposition.
A fuller self-consistency rationale will follow, but in short gap width is floored to the next lower power of two,
\begin{align*}
  2^{\lfloor \log_{2}\left(\frac{n}{r}\right) \rfloor}.
\end{align*}
The next iteration repeats the procedure from the newly retained deposition time instead of deposition time zero.
Iteration continues until reaching the newest stratum.

The set of strata to eliminate can be calculated from set subtraction between enumerations of the historical record at time points $t-1$ and $t$.
So, update time complexity follows from extant record enumeration time complexity, which turns out to be $O(\log n)$ as detailed in Theorem \ref{thm:recency-proportional-resolution-algo-uncertainty-bound}.
We provide a tested, but unproven, constant-time pruning enumeration implementation in the \texttt{hstrat} library, but will not cover it here.
The residuum order of growth $O(\log n)$ also follows closely from the record enumeration algorithm, detailed in Theorem \ref{thm:recency-proportional-resolution-algo-enum-retained-ranks}.

Why does flooring step sizes to a binary power ensure self-consistency?
Let us begin by noting properties applicable to all layers $l$,
\begin{enumeration}
\item gap width provided at retained layer $l$ increases monotonically as further depositions proceed,
\item the retained stratum preceding or at $l$ has deposition time at an even multiple of gap width (on either side), and
\item all deposition time multiples of gap width up to the newest stratum exist among retained strata.
\end{enumeration}
Observe also that gap width decreases monotonically with layer youth.

Properties 2 and 3 occur as a result of stacking monotonically-decreasing powers of two.
Subsequent smaller powers of two stack to all multiples of a larger power of 2.
Conversely, preceding larger powers of 2 can be evenly divided by succeeding smaller powers of 2, ensuring all gap edges occur at even multiples of gap width.

Under the binary flooring procedure, when gap size increases at a layer it will double (or quadruple, octuple, etc.).
Availability of the new gap endpoint is guaranteed from the stacking properties due to that endpoint being an even multiple of original step size.
Theorem \ref{thm:recency-proportional-resolution-algo-self-consistency} uses inductive proof-by-contradiction to show this point.

The recency-proportional resolution policy algorithm provides stable relative accuracy indefinitely.
This makes it particularly attractive in scenarios where phylogenetic reconstruction accuracy is paramount.
At comparable annotation sizes, we have found that recency-proportional gap width distribution outperforms uniform gap width distribution in phylogenetic information recovery \citep{moreno2022hereditary}.
Maintaining 3\% relative precision appears sufficient to eliminate most bias from reconstruction error on phylogenetic metrics \citep{moreno2023toward}

The recency-proportional policy's indefinite stability may be particularly useful in scenarios of indefinite or indeterminate record keeping duration.
Although annotation residuum size grows unboundedly, logarithmic memory usage growth is manageable in most practical scenarios.
However, this policy would not suit applications with hard caps on annotation size (e.g., fixed memory footprint digital genomes).

\begin{algorithm}
\caption{Recency-proportional Resolution Stratum Discard Generator}
\label{alg:recency-proportional-algo-gen-drop-ranks}
\begin{algorithmic}
    \Require{ $\texttt{n}$ -- the number of strata deposited }
    \Require{ $\texttt{r}$ -- the fixed resolution desired }
    \Ensure{ array of dropped strata }

    \Procedure{NumberToCondemn}{$\texttt{n}, \texttt{r}$}
        \If{$(\texttt{n} \bmod 2 = 1) \lor (\texttt{n} < 2 \cdot \texttt{r} + 1)$}
            \Return $0$
        \Else
            \Return $1 + \Call{NumberToCondemn}{$\texttt{n} / 2$, \texttt{r}}$
        \EndIf
    \EndProcedure

    \State $\texttt{num\_to\_condemn} \gets \Call{NumberToCondemn}{\texttt{n}, \texttt{r}}$
    \State $\texttt{arr} \gets \text{empty array of length num\_to\_condemn}$

    \For{$i = 0$ \textbf{to} $\texttt{num\_to\_condemn} - 1$}
        \State $\texttt{arr} [$i$] \gets \texttt{n} - 2^{i} \cdot (2 \texttt{r} + 1)$
    \EndFor
\end{algorithmic}
\end{algorithm}

\begin{algorithm}
\caption{Recency-proportional Stratum Retention Predicate}
\label{alg:recency-proportional-resolution-algo-pred-keep-rank}
\begin{algorithmic}[1]
    \STATE $\text{resolution} \gets \text{spec.GetRecencyProportionalResolution()}$
    \STATE $\text{cur\_rank} \gets 0$
    \STATE $last\_rank \gets \texttt{num\_\text{strata}\_\text{deposited}} - 1$
    \WHILE{$last\_rank - \text{cur\_rank} > \text{resolution}$}
        \STATE \textbf{yield} $\text{cur\_rank}$
        \STATE $\text{cur\_rank} \gets \text{cur\_rank} + \texttt{calc\_provided\_uncertainty(} \text{resolution}, \text{last\_rank - cur\_rank} \texttt{)}$
    \ENDWHILE
    \FOR{$i = \max(\text{last\_rank} - \text{resolution}, 0)$ \textbf{to} $\texttt{num\_\text{strata}\_\text{deposited}} - 1$}
        \STATE \textbf{yield} $i$
    \ENDFOR
\end{algorithmic}
\end{algorithm}

\begin{algorithm}
\caption{Recency Proportional Resolution Stratum Enumeration}
\label{alg:recency-proportional-resolution-algo-enum-retained-ranks}
\begin{algorithmic}[1]
    \Require{ \colorT -- the number of strata deposited }
    \State $\colorTbar \gets 0$
    \While{$\colorTbar < \colorT$}
        \State \textbf{yield} $\colorTbar$
        \State $\colorTbar \gets \colorTbar + \max\Big(\left \lfloor \frac{\colorT  - 1 - \colorTbar}{r + 1} \right \rfloor_{\mathrm{bin}}, \;\; 1\Big)$
    \EndWhile
\end{algorithmic}
\end{algorithm}


\begin{theorem}{Recency-proportional Algorithm Self-Consistency}
\label{thm:recency-proportional-algo-self-consistency}
Let $\mathsf{RPR\_kept}(n)$ be the set of data points retained at time point $n$, defined per Equation \ref{eqn:dpr_kept}.
Showing self-consistency requires that we demonstrate $\forall m \geq 0, n \in [0 \twodots m)$,
\begin{align*}
\mathsf{RPR\_kept}(n)
\geq
\mathsf{RPR\_kept}(m)
\setminus
\mathsf{future}(n, m).
\end{align*}
where $\mathsf{future}(n, m) = [n \twodots m - 1)$.
\end{theorem}

\begin{proof}
\label{prf:recency-proportional-algo-self-consistency}
Our goal is equivalent to showing that $\forall m \geq 0, n \in [0 \twodots m)$,
\begin{align*}
\mathsf{RPR\_kept}(n) \cup \mathsf{future}(n, m) \geq \mathsf{RPR\_kept}(m).
\end{align*}

Let
\begin{align*}
R_m
&=
\max\Big(\left \lfloor \frac{m  - 1 - \colorTbar}{r + 1} \right \rfloor_{\mathrm{bin}}, \;\; 1\Big)\\
R_n
&=
\max\Big(\left \lfloor \frac{n  - 1 - \colorTbar}{r + 1} \right \rfloor_{\mathrm{bin}}, \;\; 1\Big).
\end{align*}
Note that we have both $R_m, R_n \in 2^{\mathbb{N}}$.
Further, because $n < m$ we have $R_n \leq R_m$.
From these two observations, we may remark that $R_n$ evenly divides $R_m$
\begin{align}
R_m \bmod R_n = 0.
\label{eqn:rmrn0a}
\end{align}

Suppose $\colorTbar \in \mathsf{RPR\_kept}(m)$.
In this case, at least one of the following is true (A) $k = m - 1$ or (B) $k \bmod R_m = 0$.

In the first case, $\colorTbar \stackrel{\checkmark}{\in} \mathsf{future}(n, m)$.
In the second case, we may conclude from Equation \ref{eqn:rmrn0a} that $\colorTbar \bmod R_n = 0$.
So, $\colorTbar \stackrel{\checkmark}{\in} \mathsf{RPR\_kept}(n)$.
\end{proof}

\begin{theorem}{Recency-proportional Resolution Space Complexity}
\label{thm:recency-proportional-resolution-algo-space-complexity}

The \gls{extant record size} of the Recency-proportional Resolution Policy Algorithm grows with order $\mathcal{theta}{(k \log{n})}.$

\end{theorem}

\begin{proof}
\label{prf:recency-proportional-resolution-algo-space-complexity}
As per \ref{sec:extant_record_oog}, we will set out to prove that output array of this policy algorithm has an order of growth of $\mathcal{theta}{(k \log{n})},$ where $k$ is a user-provided resolution and $n$ is the number of depositions.

Algorithm \ref{alg:recency-proportional-algo-gen-drop-ranks} determines the array of strata to be dropped at any given time point.
Observe that whenever $2 \mid n$ at least one stratum will be dropped.
More generally, for any positive integer $i \le log_2{n}$, we have that if $2^i \mid n$ then $i$ strata will be dropped. 
Thus, the number of dropped strata is bounded above by $\sum_{i=1}^{log_2{n}} n = n \log_2{n}.$
As such, the number of retained strata is bound by $n - n \log_2{n} \le \log_2{n}$ for all positive $n.$
Given that no strata are dropped when $\frac{n}{2} - 1 < k,$ we observe that the output array of this policy algorithm is bound above by $\mathcal{O}{(k \log{n})}.$
Via \ref{sec:extant_record_oog}, we can conclude that this bound is actually $\mathcal{theta}{(k \log{n})}.$
\end{proof}

% TODO: missing proof
\begin{theorem}{Recency-proportional Resolution Uncertainty Bound}
\label{thm:recency-proportional-resolution-algo-uncertainty-bound}

\end{theorem}

\begin{proof}
\label{prf:recency-proportional-resolution-algo-uncertainty-bound}

\end{proof}

