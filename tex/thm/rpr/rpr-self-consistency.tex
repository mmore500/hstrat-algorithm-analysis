\begin{theorem}{Recency-proportional Algorithm Self-Consistency}
\label{thm:recency-proportional-algo-self-consistency}
Let $\mathsf{RPR\_kept}(n)$ be the set of data points retained at time point $n$, defined per Equation \ref{eqn:dpr_kept}.
Showing self-consistency requires that we demonstrate $\forall m \geq 0, n \in [0 \twodots m)$,
\begin{align*}
\mathsf{RPR\_kept}(n)
\geq
\mathsf{RPR\_kept}(m)
\setminus
\mathsf{future}(n, m).
\end{align*}
where $\mathsf{future}(n, m) = [n \twodots m - 1)$.
\end{theorem}

\begin{proof}
\label{prf:recency-proportional-algo-self-consistency}
Our goal is equivalent to showing that $\forall m \geq 0, n \in [0 \twodots m)$,
\begin{align*}
\mathsf{RPR\_kept}(n) \cup \mathsf{future}(n, m) \geq \mathsf{RPR\_kept}(m).
\end{align*}

Let
\begin{align*}
R_m
&=
\max\Big(\left \lfloor \frac{m  - 1 - \colorTbar}{r + 1} \right \rfloor_{\mathrm{bin}}, \;\; 1\Big)\\
R_n
&=
\max\Big(\left \lfloor \frac{n  - 1 - \colorTbar}{r + 1} \right \rfloor_{\mathrm{bin}}, \;\; 1\Big).
\end{align*}
Note that we have both $R_m, R_n \in 2^{\mathbb{N}}$.
Further, because $n < m$ we have $R_n \leq R_m$.
From these two observations, we may remark that $R_n$ evenly divides $R_m$
\begin{align}
R_m \bmod R_n = 0.
\label{eqn:rmrn0a}
\end{align}

Suppose $\colorTbar \in \mathsf{RPR\_kept}(m)$.
In this case, at least one of the following is true (A) $k = m - 1$ or (B) $k \bmod R_m = 0$.

In the first case, $\colorTbar \stackrel{\checkmark}{\in} \mathsf{future}(n, m)$.
In the second case, we may conclude from Equation \ref{eqn:rmrn0a} that $\colorTbar \bmod R_n = 0$.
So, $\colorTbar \stackrel{\checkmark}{\in} \mathsf{RPR\_kept}(n)$.
\end{proof}
