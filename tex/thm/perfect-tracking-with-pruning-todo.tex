\begin{theorem}{Pruning Time Complexity}
\label{thm:perfect-tracking-with-pruning-todo}
The time complexity of pruning in perfect phylogenetic tracking is $\mathcal{O}(1)$, amortized. 
\end{theorem}

\begin{proof}
\label{prf:perfect-tracking-with-pruning-todo}
The added time complexity of pruning is incurred when taxa are removed.
When taxon $t$ is removed, there are two possibilities: 1) $t$'s parent is either alive or has other descendants that are alive, or 2) $t$ was the last living descendant of it's parent.
In case 1, nothing other than $t$ can be removed from the phylogeny.
Consequently, case 1 takes constant time, as the only added cost from pruning is checking whether we are in case 1 or case 2.

In case 2, pruning must be done.
We recurse back up the lineage until we find a taxon, $a$, that is either alive or has living descendants. 
Let the distance between $t$ and the root of the tree be $d_t$ and the distance between $a$ and the root of the tree be $d_a$.
The pruning operation, then, take $\mathcal{O}(d_t - d_a)$ steps. 
In the worst case, this value will be equal to the number of elapsed generations.

However, for case 2 to occur, all taxa from $t$ to $a$ (including $a$ but not including $t$) must have already been removed.
Consequently, $d_t - d_a$ case 1 removal operations must happen for every case 2 removal operation.
Thus, the amortized time to remove the sequence of taxa from $t$ to $a$ (including both $t$ and $a$) is:

\[
\frac{2(d_t - d_a) + 1}{d_t - d_a + 1}
\]

Since

\[
\lim_{d_t - d_a\to\infty} \frac{2(d_t - d_a) + 1}{d_t - d_a + 1} = 2
\],

the amortized time complexity of pruning is bounded by a constant, i.e. it is $\mathcal{O}(1)$.


\end{proof}
